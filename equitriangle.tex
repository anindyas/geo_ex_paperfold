%%%%%%%%%%%%%%%%%%%%%%%%%%%%%%%%%%%%%%%%%%%%%%%%%%%%%%%%%%%%%%%%%%%%%%%%%%%%%%%%

\chapter{THE EQUILATERAL TRIANGLE.}


\begin{enumerate}

    \item Now take this square piece of paper (Fig. 9), and fold it double,
        laying two opposite edges one upon the other. We obtain a crease which
        passes through the mid-points of the remaining sides and is at right
        angles to those sides.  Take any point on this line, fold through it and
        the two corners of the square which are on each side of it. We thus get
        isosceles triangles standing on a side of the square.
    
%Fig. 9.  Fig. 10.  Fig. 11.

    \item The middle line divides the isosceles triangle into two congruent
        right-angled triangles.
    
    \item The vertical angle is bisected.
    
    \item If we so take the point on the middle line, that its distances from
        two corners of the square are equal to a side of it, we shall obtain an
        equilateral triangle (Fig. 10).  This point is easily determined by
        turning the base $A$ through one end of it, over $AA'$, until the other
        end, $B$, rests upon the middle line, as at $C$.
    
    \item Fold the equilateral triangle by laying each of the sides upon the
        base.  We thus obtain the three altitudes of the triangle, viz.: $AA'$,
        $BB'$, $CC'$, (Fig.  11).
    
    \item Each of the altitudes divides the triangle into two congruent
        right-angled triangles.

    \item They bisect the sides at right angles.

    \item They pass through a common point.

    \item Let the altitudes $AA'$ and $CC'$ meet in $O$.  Draw $BO$ and produce
        it to meet $AC$ in $B'$.  $BB'$ will now be proved to be the third
        altitude.  From the triangles $C'OA$ and $COA'$, $OC'=OA'$.  From
        triangles $OC'B$ and $A'OB$, $\angle OBC'= \angle A'BO$.  Again from
        triangles $ABB'$ and $CB'B$, $\angle AB'B = \angle BB'C$, i.e., each of
        them is a right angle.  That is, $BOB'$ is an altitude of the
        equilateral triangle $ABC$.  It also bisects $AC$ in $B'$.

    \item It can be proved as above that $OA$, $OB$, and $OC$ are equal, and
        that $OA'$, $OB'$, and $OC'$ are also equal.

    \item Circles can therefore be described with $O$ as a center and passing
        respectively through $A$, $B$, and $C$ and through $A'$, $B'$, and $C$.
        The latter circle touches the sides of the triangle.
    
    \item The equilateral triangle $ABC$ is divided into six congruent
        right-angled triangles which have one set of their equal angles at $O$,
        and into three congruent, symmetric, concyclic quadrilaterals.

    \item The triangle $AOC$ is double the triangle $A'OC$; therefore, $AO = 2
        OA'$.  Similarly, $BO=2OB'$ and $CO=2OC'$. Hence the radius of the
        circumscribed circle of triangle ABC is twice the radius of the
        inscribed circle.
    
    \item The right angle $A$, of the square, is trisected by the straight lines
        $AO$, $AC$.  Angle $BAC = \frac{2}{3}$ of a right angle. The angles
        $C'AO$ and $OABD'$ are each $\frac{1}{3}$ of a right angle.  Similarly
        with the angles at $B$ and $C$.

    \item The six angles at $O$ are each $\frac{2}{3}$ of a right angle.
    
    \item Fold through $A'B'$, $BC'$, and $C'A'$ (Fig. 12).  Then $A'B'C'$ is an
        equilateral triangle.  It is a fourth of the triangle $ABC$.
    
    \item $A'B'$, $B'C'$, $C'A'$ are each parallel to $AB$, $BC$, $CA$, and
        halves of them.
    
    \item $AC'A'B'$ is a rhombus. So are $C'BA'B'$ and $CB'C'A'$.
    
    \item $A'B'$, $B'C'$, $C'A'$ bisect the corresponding altitudes.
    
    \item $CC'^{2} + AC'^{2} =  CC'^{2} + \frac{1}{4} AC^{2} = AC^{2}$\\
        $\therefore CC'^{2} = \frac{3}{4} AC^{2}$\\
        $\therefore CC' = \frac{1}{2} \sqrt{3} \cdot AC = \frac{1}{2} \sqrt{3}
        \cdot AB $\\
        $ = 0.866\ldots \times AB$.
    
    %Fig. 12.
    
    \item The $\triangle ABC =$ rectangle of $AC'$ and $CC'$, i.e.\ $\frac{1}{2}
        AB \times \frac{1}{2} \sqrt{3} \cdot AB = \frac{1}{4} \sqrt{3} \cdot
        AB^{2} = 0.433\ldots \times AB^{2}$.
    
    \item The angles of the triangle $AC'C$ are in the ratio of $1:2:3$, and its
        sides are in the ratio of $\sqrt{1}:\sqrt{3}:\sqrt{4}$.

\end{enumerate}

%%%%%%%%%%%%%%%%%%%%%%%%%%%%%%%%%%%%%%%%%%%%%%%%%%%%%%%%%%%%%%%%%%%%%%%%%%%%%%%%
