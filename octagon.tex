%%%%%%%%%%%%%%%%%%%%%%%%%%%%%%%%%%%%%%%%%%%%%%%%%%%%%%%%%%%%%%%%%%%%%%%%%%%%%%%%

\chapter{THE OCTAGON.}


\begin{enumerate}

    \item To cut off a regular octagon from a given square.  Obtain the
        inscribed square by joining the mid-points $A,$, $B,$, $C,$, $D$ of the
        sides of the given square.

%Fig. 32.

        Bisect the angles which the sides of the inscribed square make with the
        sides of the other. Let the bisecting lines meet in Z. /, G, and ZH.

        $AEBFCGDA$ is a regular octagon.
    
        The triangles $AEB$, $BFC$, $CGD$, and $DHA$ are congruent isosceles
        triangles.  The octagon is therefore equilateral.
    
        The angles at the vertices, Z, 7, G, H of the same four triangles are
        each one right angle and a half, since the angles at the base are each
        one-fourth of a right angle.
    
        Therefore the angles of the octagon at A, B, C, and D are each one right
        angle and a half.
    
        Thus the octagon is equiangular.
    
        The greatest breadth of the octagon is the side of the given square, a.
    
    \item If \& be the radius of the circumscribed circle, and a be the side of
        tke original square, a
    
    \item The angle subtended at the center by each of the sides is half a right
        angle.
    
    
    \item Draw the radius $OB$ and let it cut $AB$ in $K$ (Fig. 33).
    
%Then AK = OK= C4 wh V 2 2V 2 rT as a oe i ae =< KE=04A—0OK=5 Rare V2).
%
%
%Now from triangle 474, AP=—AK%+ KF @
%
%
%Ee eee ooo 2/2 =3 1 3) Soe
%
%a — = g (4—2V2) a =~
%
%S
%
%
%AB= 5 VY 2—V2.
%
%
%g1. The altitude of the octagon is CZ (Fig. 33).  But CH? —=AC*—_ AF a > a =
        %=@_— 7+ Q—V2)=7-2+V2).  c
%
%
%DS MDE
%
%
%ANG
%
%
%Fig. 33.
%
%
%eae lt Be Wa
%
%
%2
%
%
    \item The area of the octagon is eight times the triangle $AOE$ and
%
%
%2 Bede feed a 2 32 v2

    \item A regular octagon may also be obtained by dividing the angles of the
        given square into four equal parts.


%<I YS yo
%
%
%Ww K Fig. 34.
%
%
%
%It is easily seen that EZ= WZ—=a, the side of the square. .  XZ=aV2;
    %XE=a(V2—1); XE=WH=-WEK: KX =a—a(V2—1) —a(2—V2).  Now KZ?=a?4+ a2(V2—1)? =a?
    %(4—2V2) . KZ=aV 4—2V2.  Also GE =XZ—2XE =aV 2—2a(V2— 1)
%
%
%=a(2—V 2).
%
%+. HO=5 (2-V2).  Again 0Z=5V2, and HZ?— HO*?+ OZ?  =" (6—4V242)
%
%
%=a? (2—/Y2).  2. AZ=et/ 2—72.
%
%
%HK = KZ— HZ at|y t-29 2 —a|/ ov 2 en(/ 2-72 )c Vv 2—1)
%
%
%=aVy10—TV2.  1 : “1 /i0_ Ty 2.  and HA=SV2—14V2.
%
%
\item The area of the octagon is eight times the area of the triangle $HOA$,
%
%
%Bra V2
%
%
%%= HO? 2/2 =|$(@—-v2) ):2v2 —7-2V2-(6—4V2)
%
%
%=a? (3V2—4) =a?-V2-(V2—1)?.
%
%
%, aS “oka re?) .  cv 5 ate ‘ 7 a? a Lar 7 7 .  or . * 44 GEOMETRIC EXERCISES
%
%
%\item This octagon: the octagonin§ 92 — = (2—V2)?:1 or 2:(V24 1); and their
    %bases are to one another as
%
%
%a vo se
%
\end{enumerate}


%%%%%%%%%%%%%%%%%%%%%%%%%%%%%%%%%%%%%%%%%%%%%%%%%%%%%%%%%%%%%%%%%%%%%%%%%%%%%%%%
