%%%%%%%%%%%%%%%%%%%%%%%%%%%%%%%%%%%%%%%%%%%%%%%%%%%%%%%%%%%%%%%%%%%%%%%%%%%%%%%%

\chapter{GENERAL PRINCIPLES.}

\begin{enumerate}

\item In the preceding pages we have adopted several processes, e.g., bisecting
      and trisecting finite lines, bisecting rectilineal angles and dividing 
      them into other equal parts, drawing perpendiculars to a given line, etc. 
      Let us now examine the theory of these processes.

\item The general principle is that of congruence.  Figures and straight lines
      are said to be congruent, if they are identically equal, or equal in all 
      respects.

      In doubling a piece of paper upon itself, we obtain the straight edges of
      two planes coinciding with each other.  This line may also be regarded as
      the intersection of two planes if we consider their position during the
      process of folding.

      In dividing a finite straight line, or an angle into a number of equal parts,
      we obtain a number of congruent parts.  Equal lines or equal angles are
      congruent.


\item Let $X'X$ be a given finite line, divided into any two parts by $A'$. Take
      $O$ the mid-point by doubling the line on itself. Then $OA'$ is half the
      difference between $A'X$ and $X'A'$. Fold $X'X$ over $O$, and take $A$ in
      $OX$ corresponding to $A'$.  Then $AJ'$ is the difference between $A'X$ 
      and $X'A'$ and it is bisected in $O$.  

%       pears | | a A’ O A xX Fig. 52.

      As $A'$ is taken nearer $O$, $A'O$ diminishes, and at the same time $A'A$
      diminishes at twice the rate. This property is made use of in finding the 
      mid-point of a line by means of the compasses.

\item The above observations apply also to an angle. The line of bisection is
      found easily by the compasses by taking the point of intersection of two 
      circles.


\item In the line X’X, segments to the right of O may be considered positive
      and segments to the left of O may be considered negative. That is, a
      point moving from O to 4 moves positively, and a point moving in the
      opposite direction OA’ moves negatively.  AX = OX— OA.


%OA OX SAX" both members of the equation being negative. * 152. If OA, one arm
        %of an angle 4 OP, be fixed and
%
%
%OP be considered to revolve round O, the angles which it makes with OA are of
        %different magnitudes.
%
%
%*See Beman and Smith's New Plane and Solid Geometry, p. 56.
%
      All such angles formed by $OF$ revolving in the direction opposite to that 
      of the hands of a watch are regarded positive. The angles formed by $OP$
      revolving in an opposite direction are regarded negative.* .


\item After one revolution, $OP$ coincides with $OA$.  Then the angle described 
    is called a perigon, which evidently equals four right angles.  When $OP$ has
    completed half the revolution, it is in a line with $OAB$.  Then the angle
    described is called a straight angle, which evidently equals two right
    angles.  When $OP$ has completed quarter of a revolution, it is 
    perpendicular to $OA$.  All right angles are equal in magnitude.  So are all
    straight angles and all perigons.


\item Two lines at right angles to each other form four congruent quadrants.
    Two lines otherwise inclined form four angles, of which those vertically
    opposite are congruent.


\item The position of a point in a plane is determined by its distance from each
    of two lines taken as above.  The distance from one line is measured 
    parallel to the other. In analytic geometry the properties of plane figures 
    are investigated by this method.  The two lines are called axes; the 
    distances of the point from the axes are called co-ordinates, and the 
    intersection of the axes is called the origin. This has greatly helped 
    modern research.


%*See Beman and Smith's New Plane and Solid Geometry, p. 56.  tZd., p. 5.
%method was invented by Descartes in 1637 A. D.* It

\item If $X'X$, $YY'$ be two axes intersecting at $O$, distances measured in the
    direction of $OX$, i.e., to the right of $O$ are positive, while distances
    measured to the left of $O$ are negative. Similarly with reference to $YY'$,
    distances measured in the direction of $OY$ are positive, while distances
    measured in the direction of $OY'$ are negative.


\item Axial symmetry is defined thus: If two figures in the same plane can be
    made to coincide by turning the one about a fixed line in the plane through
    a straight angle, the two figures are said to be symmetric with regard to
    that line as axis of symmetry.


\item Central symmetry is thus defined: If two figures in the same plane can be
    made to coincide by turning the one about a fixed point in that plane
    through a straight angle, the two figures are said to be symmetric with
    regard to that point as center of symmetry.

    In the first case the revolution is outside the given plane, while in the
    second it is in the same plane.

    If in the above two cases, the two figures are halves of one figure, the
    whole figure is said to be symmetric with regard to the axis or center—these
    are called axis or center of symmetry or simply axis or center.


%*Beman and Smith’s translation of Fink’s History of Mathematics, p. 230.
%%+Beman and Smith’s New Plane and Solid Geometry, p. 26.  $/6., p. 183.

\item Now, in the quadrant $XOY$ make a triangle $PQKR$. Obtain its image in the
    quadrant $YOX'$ by folding on the axis $YY'$ and pricking through the paper 
    at the vertices. Again obtain images of the two triangles in the fourth and
    third quadrants. It is seen that the triangles in adjacent quadrants posses
    axial symmetry, while the triangles in alternate quadrants possess central 
    symmetry.

%Fig. 53.
%

\item Regular polygons of an odd number of sides possess axial symmetry, and
    regular polygons of an even number of sides possess central symmetry as
    well.


\item If a figure has two axes of symmetry at right angles to each other, the
    point of intersection of the axes is a center of symmetry.  This obtains in
    regular polygons of an even number of sides and certain curves, such as the
    circle, ellipse, hyperbola, and the lemniscate; regular polygons of an odd
    number of sides may have more axes than one, but no two of them will be at
    right angles to each other.  If a sheet of paper is folded double and cut, 
    we obtain a piece which has axial symmetry, and if it is cut fourfold, we
    obtain a piece which has central symmetry as well, as in Fig. 54.

%Fig. 54.


\item Parallelograms have a center of symmetry.  A quadrilateral of the form of
    a kite, or a trapezium with two opposite sides equal and equally inclined to
    either of the remaining sides, has an axis of symmetry.


\item The position of a point in a plane is also determined by its distance from
    a fixed point and the inclination of the line joining the two points to a
    fixed line drawn through the fixed point.

    If $OA$ be the fixed line and $P$ the given point, the length $OP$ and $AOP$,
    determine the position of $P$.

%
%Pp
%
%
%O A
%
%
%Fig. 55.
%
%
%O is called the pole, OA the prime-vector, OP the radius vector, and / AOP the
        %vectorial angle. OP and / AOP are called polar co-ordinates of P.
%
%
%\item The image of a figure symmetric to the axis OA may be obtained by folding
        %through the axis OA.  The radii vectores of corresponding points are
        %equally inclined to the axis.
%
%
%\item Let ABC be a triangle. Produce the sides CA, AB, BC to D, £, F
        %respectively. Suppose a person to stand at 4 with face towards D and
        %then to
%
%proceed from 4 to B, Bto C, and Cto 4. Then he successively describes the
        %angles DAB, EBC, FCD.  Having come to his original position 4, he has
        %com-
%
%Fig. 56.
%
%
%pleted a perigon, i. e., four right angles. We may therefore infer that the
        %three exterior angles are together equal to four right angles.
%
%
%The same inference applies to any convex polygon.
%
%
%\item Suppose the man to stand at 4 with his face towards C, then to turn in
        %the direction of 48 and proceed along 4B, BC, and C4.
%
%In this case, the man completes a straight angle, i. e., two right angles. He
        %successively turns through the angles CAB, EBC, and FCA. Therefore /
        %HBF +/FCA-+ 2 CAB (neg. angle) = a straight angle.
%
%This property is made use of in turning engines on the railway. An engine
        %standing upon DA with its head towards 4 is driven on to CF, with its
        %head towards #. The motion is then reversed and it goes backwards to
        %ZB. Then it moves forward along BA on to AD. The engine has
        %successively described the angles 4CB, CBA, and BAC. Therefore the
        %three interior angles of a triangle are
%together equal to two right angles.
%
%
%\item The property that the three interior angles of a triangle are together
        %equal to two right angles is illustrated as follows by paper folding.
%
%Fold CC’ perpendicular to 42. Bisect C’A in J, and AC’ in M@. Fold V4’, 8B’
        %perpendicular to AB, meeting BC and AC in A’ and B’. Draw A'C’, BC".
%
%
%Cc
%
%Fig. 57.
%
%
%By folding the corners on V4’, AZB’ and A'S’, we find that the angles 4, B, C
        %of the triangle are equal to the angles B’C'A, BC'A’', and A'C'S’
        %respectively, which together make up two right angles.
%
%
%\item Take any line ABC. Draw perpendiculars to ABC at the points A, B, and C.
        %Take points D, &, Fin the respective perpendiculars equidistant from
        %their feet.  Then it is easily seen by superposition and proved by
        %equal triangles that DZ is equal to AB and perpendicular to AD and BZ,
        %and that EF is equal to BC and
%perpendicular to BZ and CF.  Now AB (= VDE) is the shortest distance between
        %the lines 4D and B#, and it is constant. Therefore 4D
%
%
%D E F
%
%
%H
%
%
%Pee |
%
%
%A B Cc Fig. 58.
%
%
%and #£ can never meet, i. e., they are parallel. Hence lines which are
        %perpendicular to the same line are parallel.
%
%The two angles BAD and £BPA are together equal to two right angles. If we
        %suppose the lines 4D and BE to move inwards about 4 and BZ, they will
        %meet and the interior angles will be less than two right angles. They
        %will not meet if produced backwards.  This is embodied in the much
        %abused twelfth postulate of Euclid’s Elements. *
%
%
%\item If 4GH be any line cutting BZ in Gand CF in H, then
%
%
%*For historical sketch see Beman and Smith's translation of Fink’s History of
        %Mathematics, p. 270.
%
%Z GAD= the alternate / AGB, ‘ each is the complement of / BAG; and Z HGE = the
        %interior and opposite / GAD.  . they are each = / AGB.  Also the two
        %angles GAD and ZGA are together equal to two right angles.
%
%
%\item Take a line 4X and mark off on it, from A, equal segments 4B, BC, CD,
        %DE....Erect perpendiculars to AZ at B, C, D, #....Leta line AF’ cut the
        %perpendiculars in B’, C’, D’, E’....Then AB, PO, OD ii: nese a eaunt,
%
%Fig. 59.  If 4B, BC, CD, DE be unequal, then AB: SC=AsB AC BC: CD=S'C’: C'D',
        %and soon.
%
%
%\item If ABCDE....be a polygon, similar polygons may be obtained as follows.
%
%Take any point O within the polygon, and draw OA, (Ob) OCLs.
%
%Take any point 4’ in OA and draw A'S’, BC’, C’D’,....parallel to 4B, BC,
        %CD....respectively.
%
%Then the polygon A’S’C'D’....will be similar to ABCD.... The polygons so
        %described around a common point are in perspective. The point O may
        %also lie outside the polygon. It is called the center of perspective.
%
%
%\item To divide a given line into 2, 3, 4, 5....equal parts. Let 4B be the
        %given line. Draw AC, BD
%
%
%B D
%
%
%+0
%
%
%po
%
%
%F E c wa Fig. 60.  at right angles to 48 on opposite sides and make AC=B8D.
        %Draw CD cutting Af in Py. Then AP, —Fah.
%
%Now produce AC and take CE—=ZF=FG....  =AC or BD. Draw DE, DF, DG....cutting 4B
        %Mite gene pe Pe ae cs
%
%Then from similar triangles,
%
%PeBiAPy eh T AL.
%
%-*. P3B:AB=BD:AF . =a haps Similarly (Ry ; PB VA BV 4; and so on.  lil tterl,
%
%
%Ons Oe Bre ae We ees
%
%
%But 4P,+ PoP, + P3Ps +....is ultimately = 4
%
%
%1 il 1 ae Ts ae fa tool.
%
%Or .  ie! 1 aera 1d gi 2.3: 33?  1 bbe oi nm na+l1 nati) Adding 1 1 1 1 pategt
        %Tt apd) ee
%
%1 1 1 1 . Teg teres eine tT a:
%
%
%The limit of Ae when 2 is o is l.
%
%
%\item The following simple contrivance may be used for dividing a line into a
        %number of equal parts.
%
%Take a rectangular piece of paper, and mark off x equal segments on each or one
        %of two adjacent sides.  Fold through the points of section so as to
        %obtain perpendiculars to the sides. Mark the points of section and the
        %corners 0, 1, 2,....”. Suppose it is required to divide the edge of
        %another piece of paper AB into x equal parts. Now place 42 so that 4 or
        %B may lie on 0, and B or A on the
%perpendicular through 2.
%
%In this case 4# must be greater than OV. But the smaller side of the rectangle
        %may be used for smaller lines.
%
%The points where 427 crosses the perpendiculars are the required points of
        %section.
%
%\item Center of mean position. Ifa line 42 contains (m-+-”) equal parts, and it
        %is divided at C so that 4C contains m of these parts and CZ contains ”
        %of them; then if from the points 4, C, B perpendiculars AD, CF, BE be
        %let fall on any line,
%
%m BE+n-AD=(m-+n):CF.
%
%Now, draw BGA parallel to ED cutting Cf in G and 4) in H. Suppose through the
        %points of division AB lines are drawn parallel to BH. These lines
%
%will divide 47 into (m+) equal parts and CG into m equal parts.  -°. 2' AH
        %=(m-+-n)-CG,
%
%and since DH and BE are each=G/F,
%
%n' HD-+-m: BE=‘m-+n2)GF.
%
%Hence, by addition
%
%m-1+ Dim BE=(m+2)GF
%
%n. AD-+-m BE=(m-+n): CF.
%
%C is called the center of mean position, or the mean center of 4 and # for the
        %system of multiples mand 2.
%
%The principle can be extended to any number of points, not ina line. Then if P
        %represent the feet of the perpendiculars on any line from 4, B, C,
        %etc., if a, 6, c....be the corresponding multiples, and if JZ be the
        %mean center
%
%aAP+éb-BP+c:CP....  w(¢+6+¢-+....)MP.
%
%If the multiples are all equal to a, we get
%
%a(AP+ BP+CP+....)=na:MP
%
%
%nm being the number of points.
%
%
\item The center of mean position of a number of points with equal multiples is
    obtained thus. Bisect the line joining any two points 4, Bin G, join Gtoa
    third point C and divide GC in H so that GY==14GC; join / toa fourth point D
    and divide HD in \& so that H{K=}HD and so on: the last point found will be
    the center of mean position of the system of points.

\item The notion of mean center or center of mean position is derived from
    Statics, because a system of material points having their weights denoted by
    a, 4, c...., and placed at A, B, C....would balance about the mean center JZ
    if free to rotate about J7 under the action of gravity. F ;

    The mean center has therefore a close relation to the center of gravity of
    Statics.


\item The mean center of three points not in a line, is the point of
    intersection of the medians of the triangle formed by joining the three
    points. This is also the center of gravity or mass center of a thin
    triangular plate of uniform density.


%\item If JZ is the mean center of the points 4, B, C, etc., for the
        %corresponding multiples a, 4, ¢, etc., and if P is any other point,
        %then a‘ AP?+6-BP?+¢:CP?-+....  =a:AM?+6-BM*+¢-CM?+....  + PM?
        %(a+~b+e+....).  Hence in any regular polygon, if O is the in-center or
        %circum-center and / is any point AP?4t BP?4+....=O0A*?+ OB
        %+....+4+n:OF?  =a2:(R?+ OP?).
%
%
%Now
%
%ABIL ACILAD 4... = 2n-R?.  Similarly
%
%BAt+ BC?+ BD'+....=2n- RF?
%
%
%C42 + CB24 CDI+.../=2n- Rt.
%
%Adding 2(AB?+ AC?+ AD? 4+....)=n-2n- RP?  . ABI+ AC? + AD? 4+.... 7? R?.
%
%
%\item The sum of the squares of the lines joining the mean center with the
        %points of the system is a minimum.
%
%If 17 be the mean center and P any other point not belonging to the system,
%
%3PA* = 2MA?+ 2PM", (where & stands for ‘‘the sum of all expressions of the
        %type”’).
%
%-, 2PA?* is the minimum when P—), i. e.,
%
%
%when FP is the mean center.


\item Properties relating to concurrence of lines and collinearity of points can
    be tested by paper folding.  Some instances are given below:

    \begin{enumerate}[(1)]

        \item The medians of a triangle are concurrent.  The common point is
            called the centroid.

        \item The altitudes of a triangle are concurrent.  The common point is
            called the orthocenter.

        \item The perpendicular bisectors of the sides of a triangle are
            concurrent.  The common point is called the circum-center.

        \item The bisectors of the angles of a triangle are concurrent.  The
            common point is called the in-center.

        \item Let $ABCD$ be a parallelogram and $P$ any point. Through $P$ draw
            $GH$ and $BF$ parallel to $BC$.


            *For treatment of certain of these properties see Beman and Smith's
            New Plane and Solid Geometry, pp. 84, 182.  and ABZ respectively.
            Then the diagonals 2G, HF, and the line DZ are concurrent.

        \item If two similar unequal rectineal figures are so placed that their
            corresponding sides are parallel, then the joins of corresponding
            corners are concurrent.  The common point is called the center of
            similarity.

        \item If two triangles are so placed that their corners are two and two
            on concurrent lines, then their corresponding sides intersect
            collinearly.  This is known as Desargues's theorem. The two
            triangles are said to be in perspective. The point of concurrence
            and line of collinearity are respectively called the center and axis
            of perspective.

        \item The middle points of the diagonals of a complete quadrilateral are
            collinear.

        \item If from any point on the circumference of the circum-circle of a
            triangle, perpendiculars are dropped on its sides, produced when
            necessary, the feet of these perpendiculars are collinear. This line
            is called Simson's line.

            Simson's line bisects the join of the orthocenter and the point from
            which the perpendiculars are drawn.

        \item In any triangle the orthocenter, circum-center, and centroid are
            collinear.

            The mid-point of the join of the orthocenter and circum-center is
            the center of the nine-points circle, so called because it passes
            through the feet of the altitudes and medians of the triangle and
            the mid-point of that part of each altitude which lies between the
            orthocenter and vertex.

            The center of the nine-points circle is twice as far from the
            orthocenter as from the centroid. This is known as Poncelet’s
            theorem.

        \item If $A$, $B$, $C$, $D$, $Z$, $F$, are any six points on a circle
            which.are joined successively in any order, then the intersections
            of the first and fourth, of the second and fifth, and of the third
            and sixth of these joins produced when necessary are collinear.  
            This is known as Pascal's theorem.

        \item The joins of the vertices of a triangle with the points of contact
            of the in-circle are concurrent. The same property holds for the
            ex-circles.

        \item The internal bisectors of two angles of a triangle, and the
            external bisector of the third angle intersect the opposite sides
            collinearly.

        \item The external bisectors of the angles of a triangle intersect the
            opposite sides collinearly.

        \item If any point be joined to the vertices of a triangle, the lines
            drawn through the point perpendicular to those joins intersect the
            opposite sides of the triangle collinearly.

        \item If on an axis of symmetry of the congruent triangles ABC, A'B'C' a
            point O be taken A’O, B’O, and C’O intersect the sides BC, CA and AB
            collinearly.

        \item The points of intersection of pairs of tangents to a circle at the
            extremities of chords which pass through a given point are collinear.
            This line is called the polar of the given point with respect to the 
            circle.

        \item The isogonal conjugates of three concurrent lines $AX$, $BX$, $CX$
            with respect to the three angles of a triangle 48C are concurrent.
            (Two lines $AX$, $AY$ are said to be isogonal conjugates with 
            respect to an angle $BAC$, when they make equal angles with its 
            bisector.)

        \item If in a triangle $ABC$, the lines $AA'$, $BB'$, $CC'$ drawn from
            each of the angles to the opposite sides are concurrent, their
            isotomic conjugates with respect to the corresponding sides are also
            concurrent. (The lines $AA'$, $AA''$ are said to be isotomic
            conjugates, with respect to the side $BC$ of the triangle $ABC$,
            when the intercepts $BA'$ and $CA''$ are equal.)

        \item The three symmedians of a triangle are concurrent. (The isogonal
            conjugate of a median $A//$ of a triangle is called a symmedian.)

    \end{enumerate}

\end{enumerate}


%%%%%%%%%%%%%%%%%%%%%%%%%%%%%%%%%%%%%%%%%%%%%%%%%%%%%%%%%%%%%%%%%%%%%%%%%%%%%%%%
