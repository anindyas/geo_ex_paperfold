%%%%%%%%%%%%%%%%%%%%%%%%%%%%%%%%%%%%%%%%%%%%%%%%%%%%%%%%%%%%%%%%%%%%%%%%%%%%%%%%

\chapter*{INTRODUCTION}

\begin{enumerate}

    \item THE \emph{idea} of this book was suggested to me by Kindergarten Gift
        No.\ VIII.\ --- Paper-folding. The gift consists of two hundred
        variously colored squares of paper, a folder, and diagrams and
        instructions for folding. The paper is colored and glazed on one side.
        The paper may, however, be of self-color, alike on both sides. In fact,
        any paper of moderate thickness will answer the purpose, but colored
        paper shows the creases better, and is more attractive. The kindergarten
        gift is sold by any dealers in school supplies; but colored paper of
        both sorts can be had from stationery dealers. Any sheet of paper can be
        cut into a square as explained in the opening articles of this book, but
        it is neat and convenient to have the squares ready cut.


    \item These exercises do not require mathematical instruments, the only
        things necessary being a pen-knife and scraps of paper, the latter being
        used for setting off equal lengths. The squares are themselves simple
        substitutes for a straight edge and a T-square.

    \item In paper-folding several important geometric processes can be effected
        much more easily than with a pair of compasses and ruler, the only
        instruments the use of which is sanctioned in Euclidean geometry; for
        example, to divide straight lines and angles into two or more equal
        parts, to draw perpendiculars and parallels to straight lines. It is,
        however, not possible in paper-folding to describe a circle, but a
        number of points on a circle, as well as other curves, may be obtained
        by other methods.  These exercises do not consist merely of drawing
        geometric figures involving straight lines in the ordinary way, and
        folding upon them, but they require an intelligent application of the
        simple processes peculiarly adapted to paper-folding. This will be
        apparent at the very commencement of this book.

    \item The use of the kindergarten gifts not only affords interesting
        occupations to boys and girls, but also prepares their minds for the
        appreciation of science and art. Conversely the teaching of science and
        art later on can be made interesting and based upon proper foundations
        by reference to kindergarten occupations. This is particularly the case
        with geometry, which forms the basis of every science and art. The
        teaching of plane geometry in schools can be made very interesting by
        the free use of the kindergarten gifts. It would be perfectly legitimate
        to require pupils to fold the diagrams with paper. This would give them
        neat and accurate figures, and impress the truth of the propositions
        forcibly on their minds. It would not be necessary to take any statement
        on trust.

        But what is now realised by the imagination and idealisation of clumsy
        figures can be seen in the concrete.  A fallacy like the following would
        be impossible.

    \item \emph{To prove that every triangle is isosceles}. Let $ABC$,
        Fig.\ ~\ref{fig:isosceles_intro}, be any triangle.  Bisect $AB$ in $Z$,
        and through $Z$ draw $ZO$ perpendicular to $AB$.  Bisect the angle $ACB$
        by $CO$.

        \begin{figure}
            \begin{center}
                \begin{tikzpicture}

    \coordinate (O) at (0,0);
    \coordinate (A) at (-3,-1);
    \coordinate (B) at (3,-1);
    \coordinate (C) at (-1.5,4);

    \draw (A) -- (B) -- (C) -- (A);
    \foreach \x in {A, B, C}
        \draw (O) -- (\x);
    
    \draw (O) -- ($(A)!(O)!(B)$);
    \draw (O) -- ($(A)!(O)!(C)$);
    \draw (O) -- ($(B)!(O)!(C)$);

    \node[label={south west:$A$}] at (A) {};
    \node[label={south east:$B$}] at (B) {};
    \node[label={above:$C$}] at (C) {};

    \node[label={north east:$X$}] at ($(B)!(O)!(C)$) {};
    \node[label={north west:$Y$}] at ($(A)!(O)!(C)$) {};
    \node[label={south:$Z$}]      at ($(A)!(O)!(B)$) {};

\end{tikzpicture}

            \end{center}
            \label{fig:isosceles_intro}
        \end{figure}


        \begin{enumerate}[(1)]

            \item If $CO$ and $ZO$ do not meet, they are parallel. Therefore
                $CO$ is at right angles to $AB$. Therefore $AC = BC$.

            \item If $CO$ and $ZO$ do meet, let them meet in $O$.  Draw $OX$
                perpendicular to $BC$ and $OY$ perpendicular to $AC$. Join $OA$,
                $OB$.  By Euclid I, 26 (B. and \&., § 88, cor. 7)* the triangles
                $YOC$ and $XOC$ are congruent; also by Euclid I, 47 and I, 8 (B.
                and S., § 156 and § 79)\footnote{These references are to Beman
                and Smith's New Plane and Solid Geometry, Boston, Ginn & Co.,,
                1899.} the triangles $AOY$ and $BOX$ are congruent.  Therefore
                $$AY + YC = BX + XC,$$ i.e., $$AC = BC$$.


                \begin{figure}
                    \label{fig:isosceles_fold}
                    \begin{center}
                        \caption{xx}
                        %% Nice shaded/framed paragraphs using tikz and framed
% Author: Jose Luis Diaz

\begin{tikzpicture}

    \coordinate (O) at (0,0);
    \coordinate (A) at (-3,-1);
    \coordinate (B) at ( 3,-1);
    \coordinate (C) at (-1.5,4);

    \draw (A) -- (B) -- (C) -- (A);
    \foreach \x in {A, B, C}
        \draw (O) -- (\x);
    
    \draw (O) -- ($(A)!(O)!(B)$);

    \node[label={below:$A$}] at (A) {};
    \node[label={below:$B$}] at (B) {};
    \node[label={above:$C$}] at (C) {};

    \node[label={north west:$Z$}] at ($(A)!(O)!(B)$) {};

\end{tikzpicture}


%%%%%%%%%%%%%%%%%%%%%%%%%%%%%%%%%%%%%%%%%%%%%%%%%%%%%%%

% define styles for the normal border and the torn border
\tikzset{
  normal border/.style={orange!30!black!10, decorate,
     decoration={random steps, segment length=2.5cm, amplitude=.7mm}},
  torn border/.style={orange!30!black!5, decorate,
     decoration={random steps, segment length=.5cm, amplitude=1.7mm}}}

% Macro to draw the shape behind the text, when it fits completly in the
% page
\def\parchmentframe#1{
\tikz{
  \node[inner sep=2em] (A) {#1};  % Draw the text of the node
  \begin{pgfonlayer}{background}  % Draw the shape behind
  \fill[normal border]
        (A.south east) -- (A.south west) --
        (A.north west) -- (A.north east) -- cycle;
  \end{pgfonlayer}}}


% Define the environment which puts the frame
% In this case, the environment also accepts an argument with an optional
% title (which defaults to ``Example'', which is typeset in a box overlaid
% on the top border
\newenvironment{parchment}[1][Example]{%
  \def\FrameCommand{\parchmentframe}%
  \def\FirstFrameCommand{\parchmentframetop}%
  \def\LastFrameCommand{\parchmentframebottom}%
  \def\MidFrameCommand{\parchmentframemiddle}%
  \vskip\baselineskip
  \MakeFramed {\FrameRestore}
  \noindent\tikz\node[inner sep=1ex, draw=black!20,fill=white,
          anchor=west, overlay] at (0em, 2em) {\sffamily#1};\par}%
{\endMakeFramed}

%%%%%%%%%%%%%%%%%%%%%%%%%%%%%%%%%%%%%%%%%%%%%%%%%%%%%%%

\newcounter{mathseed}
\setcounter{mathseed}{3}
\pgfmathsetseed{\arabic{mathseed}} % To have predictable results
% Define a background layer, in which the parchment shape is drawn
\pgfdeclarelayer{background}
\pgfsetlayers{background,main}

% This is the base for the fractal decoration. It takes a random point between
% the start and end, and raises it a random amount, thus transforming a segment
% into two, connected at that raised point This decoration can be applied again
% to each one of the resulting segments and so on, in a similar way of a Koch
% snowflake.
\pgfdeclaredecoration{irregular fractal line}{init}
{
  \state{init}[width=\pgfdecoratedinputsegmentremainingdistance]
  {
    \pgfpathlineto{\pgfpoint{random*\pgfdecoratedinputsegmentremainingdistance}{(random*\pgfdecorationsegmentamplitude-0.02)*\pgfdecoratedinputsegmentremainingdistance}}
    \pgfpathlineto{\pgfpoint{\pgfdecoratedinputsegmentremainingdistance}{0pt}}
  }
}


% define some styles
\tikzset{
   
    paper/.style={draw=black!10, blur shadow, 
                  every shadow/.style={opacity=1, black}, 
                  shade=bilinear interpolation, 
                  lower left=black!10, upper left=black!5, 
                  upper right=white, lower right=black!5, 
                  fill=none},

    irregular cloudy border/.style={decoration={irregular fractal line, 
                                    amplitude=0.2}, decorate,},

    irregular spiky border/.style={decoration={irregular fractal line, 
                                   amplitude=-0.2}, decorate,},

    ragged border/.style= {decoration={random steps, segment length=7mm, 
                                       amplitude=2mm}, decorate,}
}

\def\tornpaper#1{

    \ifthenelse{\isodd{\value{mathseed}}}{%
    \tikz{
      \node[inner sep=1em] (A) {#1};  % Draw the text of the node

      \begin{pgfonlayer}{background}  % Draw the shape behind

      \fill[paper] % recursively decorate the bottom border
         \pgfextra{\pgfmathsetseed{\arabic{mathseed}}\addtocounter{mathseed}{1}}
          {decorate[irregular cloudy border]{decorate{decorate{decorate{decorate[ragged border]{
            (A.north west) -- (A.north east)
          }}}}}}
          -- (A.south east)
         \pgfextra{\pgfmathsetseed{\arabic{mathseed}}}%
          {decorate[irregular spiky border]{decorate{decorate{decorate{decorate[ragged border]{
          -- (A.south west)
          }}}}}}
          -- (A.north west);

      \end{pgfonlayer}}
    }

    {
    \tikz{
      \node[inner sep=1em] (A) {#1};  % Draw the text of the node

      \begin{pgfonlayer}{background}  % Draw the shape behind

      \fill[paper] % recursively decorate the bottom border
         \pgfextra{\pgfmathsetseed{\arabic{mathseed}}\addtocounter{mathseed}{1}}%
          {decorate[irregular spiky border]{decorate{decorate{decorate{decorate[ragged border]{
            (A.north east) -- (A.north west)
          }}}}}}
          -- (A.south west)
         \pgfextra{\pgfmathsetseed{\arabic{mathseed}}}%
          {decorate[irregular cloudy border]{decorate{decorate{decorate{decorate[ragged border]{
          -- (A.south east)
          }}}}}}
          -- (A.north east);

      \end{pgfonlayer}}
    }
}

%\tornpaper{ \parbox{.9\textwidth}{\lipsum[11]} }

%\begin{tikzpicture}
%
%    \begin{axis}[colormap/blackwhite, view={30}{60}, axis lines=none]
%
%       \addplot3[surf,shader=interp, samples=60, point meta=x+3*z*z-0.25*y,
%                 domain=0:2*pi,y domain=0:2*pi, z buffer=sort]
%       ({(2+cos(deg(x)))*cos(deg(y))},
%        {(2+cos(deg(x)))*sin(deg(y))},
%        {sin(deg(x))});
%
%   \end{axis}
%
%\end{tikzpicture}

                    \end{center}
                \end{figure}


                Fig.~\ref{fig:isosceles_fold} shows by paper-folding that,
                whatever triangle be taken, $CO$ and $ZO$ cannot meet within the
                triangle.

                $O$ is the mid-point of the arc $AOB$ of the circle which
                circumscribes the triangle $ABC$.

        \end{enumerate}

    \item Paper-folding is not quite foreign to us. Folding paper squares into
        natural objects --- a boat, double boat, ink bottle, cup-plate, etc., is
        well known, as also the cutting of paper in symmetric forms for purposes
        of decoration. In writing Sanskrit and Mahrati, the paper is folded
        vertically or horizontally to keep the lines and columns straight.  In
        copying letters in public offices, an even margin is secured by folding
        the paper vertically.  Rectangular pieces of paper folded double have
        generally been used for writing, and before the introduction of
        machine-cut letter paper and envelopes of various sizes, sheets of
        convenient size were cut by folding and tearing larger sheets, and the
        second half of the paper was folded into an envelope inclosing the first
        half.  This latter process saved paper and had the obvious advantage of
        securing the post marks on the paper written upon.  Paper-folding has
        been resorted to in teaching the XIth Book of Euclid, which deals with
        figures of three dimensions\footnote{See especially Beman and Smith’s
        New Plane and Solid Geometry, p, 287.}.  But it has seldom been used in
        respect of plane figures.


    \item I have attempted not to write a complete treatise or text-book on
        geometry, but to show how regular polygons, circles and other curves can
        be folded or pricked on paper. I have taken the opportunity to introduce
        to the reader some well known problems of ancient and modern geometry,
        and to show how algebra and trigonometry may be advantageously applied
        to geometry, so as to elucidate each of the subjects which are usually
        kept in separate pigeon-holes.


    \item The first nine chapters deal with the folding of the regular polygons
        treated in the first four books of Euclid, and of the nonagon.  The
        paper square of the kindergarten has been taken as the foundation, and
        the other regular polygons have been worked out thereon.  Chapter I
        shows how the fundamental square is to be cut and how it can be folded
        into equal right-angled isosceles triangles and squares.  Chapter II
        deals with the equilateral triangle described on one of the sides of the
        square.  Chapter III is devoted to the Pythagorean theorem (B. and S., §
        156) and the propositions of the second book of Euclid and certain
        puzzles connected therewith.  It is also shown how a right-angled
        triangle with a given altitude can be described on a given base.  This
        is tantamount to finding points on a circle with a given diameter.

    \item Chapter X deals with the arithmetic, geometric, and harmonic
        progressions and the summation of certain arithmetic series.  In
        treating of the progressions, lines whose lengths form a progressive
        series are obtained.  A rectangular piece of paper chequered into
        squares exemplifies an arithmetic series.  For the geometric the
        properties of the right-angled triangle, that the altitude from the
        right angle is a mean proportional between the segments of the
        hypotenuse (B. and.S., § 270), and that either side is a mean
        proportional between its projection on the hypotenuse and the
        hypotenuse, are made use of. In this connexion the Delian problem of
        duplicating a cube has been explained\footnote{See Beman and Smith’s
        translation-of Klein's Famous Problems of Elementary Geometry, Boston,
        1897; also their translation of Fink’s History of Mathematics, Chicago,
        The Open Court Pub. Co., 1900.}  In treating of harmonic progression,
        the fact that the bisectors of an interior and corresponding exterior
        angle of a triangle divide the opposite side in the ratio of the other
        sides of the triangle (B.  and S., § 249) has been used.  This affords
        an interesting method of graphically explaining systems in involution.
        The sums of the natural numbers and of their cubes have been obtained
        graphically, and the sums of certain other series have been deduced
        therefrom.

    \item Chapter XI deals with the general theory of regular polygons, and the
        calculation of the numerical value of $\pi$. The propositions in this
        chapter are very interesting.

    \item Chapter XII explains certain general principles, which have been made
        use of in the preceding chapters, --- congruence, symmetry, and
        similarity of figures, concurrence of straight lines, and collinearity
        of points are touched upon.

    \item Chapters XIII and XIV deal with the conic sections and other
        interesting curves. As regards the circle, its harmonic properties among
        others are treated.  The theories of inversion and co-axial circles are
        also explained.  As regards other curves it is shown how they can be
        marked on paper by paper-folding.  The history of some of the curves is
        given, and it is shown how they were utilised in the solution of the
        classical problems, to find two geometric means between two given lines,
        and to trisect a given rectilineal angle.  Although the investigation of
        the properties of the curves involves a knowledge of advanced
        mathematics, their genesis is easily understood and is interesting.

    \item I have sought not only to aid the teaching of geometry in schools and
        colleges, but also to afford mathematical recreation to young and old,
        in an attractive and cheap form. ``Old boys'' like myself may find the
        book useful to revive their old lessons, and to have a peep into modern
        developments which, although very interesting and instructive, have been
        ignored by university teachers.

\end{enumerate}

\begin{flushright}
{\large T. Sundara Row.\mbox\qquad}\\
\end{flushright}
Madras, India, 1893.

%%%%%%%%%%%%%%%%%%%%%%%%%%%%%%%%%%%%%%%%%%%%%%%%%%%%%%%%%%%%%%%%%%%%%%%%%%%%%%%%
