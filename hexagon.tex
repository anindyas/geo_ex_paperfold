%%%%%%%%%%%%%%%%%%%%%%%%%%%%%%%%%%%%%%%%%%%%%%%%%%%%%%%%%%%%%%%%%%%%%%%%%%%%%%%%

\chapter{THE HEXAGON.}


\begin{enumerate}

\item To cut off a regular hexagon from a given square.  Fold through the
    mid-points of the opposite sides, and obtain the lines 408 and COD.

%Fig. 29.

    On both sides of $AO$ and $OB$, describe equilateral triangles (§ 25),
    $AOL$, $AHO$; $BFO$ and $BOG$.

    Draw $EF$ and $HG$.

    $AHGSFE$ is a regular hexagon.

    It is unnecessary to give the proof.

    The greatest breadth of the hexagon is 4.


\item The altitude of the hexagon is
%
%
%V 3 BEER EE
%
%
%Fig. 30.
%

\item If R be the radius of the circumscribed circle,


%1 R=; AB.
%

\item If be the radius of the inscribed circle,


%3 ra! AB=0.433....X AB,


\item The area of the hexagon is 6 times the area of the triangle HGO,


%AB V3 ae 1s -AB?=0.6495....< AB?  Also the hexagon = 3: AB:CD.


%= 1} times the equilateral triangle on 4B.

%Fig. 31.



\item Fig. 30 is an example of ornamental folding into equilateral triangles and
    hexagons.

\item A hexagon is formed from an equilateral triangle by folding the three
    corners to the center.  The side of the hexagon is 4 of the side of the
    equilateral triangle.

    The area of the hexagon = 3 of the equilateral triangle.

\item The hexagon can be divided into equal regular hexagons and equilateral
    triangles as in Fig. 31 by folding through the points of trisection of the
    sides.

\end{enumerate}

%%%%%%%%%%%%%%%%%%%%%%%%%%%%%%%%%%%%%%%%%%%%%%%%%%%%%%%%%%%%%%%%%%%%%%%%%%%%%%%%
