%%%%%%%%%%%%%%%%%%%%%%%%%%%%%%%%%%%%%%%%%%%%%%%%%%%%%%%%%%%%%%%%%%%%%%%%%%%%%%%%

\chapter{THE PENTEDECAGON.}


\begin{enumerate}

\item Fig. 39 shows how the pentedecagon is obtained from the pentagon.  Let
    $ABCDE$ be the pentagon and $QO$ its center.

%
%
%JR
%
%Dal VAY TIES
%
%
%Fig. 39.

    Draw $OA$, $OB$, $OC$, $OD$, and $OZ$. Produce $DO$ to meet $AZ$ in $K$.

%Take OF =} of OD.

    Fold $GFA$ at right angles to $OF$. Make $OG= OH = OD$.

    Then $GDA$ is an equilateral triangle, and the angles $DOG$ and $HOD$ are
    each $120°$

    But angle $DOA$ is $144°$; therefore angle $GOA$ is $24°$,

    That is, the angle $HOA$, which is $72°$, is trisected by $OG$.

    Bisect the angle $HOG$ by $OZ$, meeting $ZA$ in $Z$, and let $OG$ cut $EA$
    in $WM$; then

%OL=OM.
%
    In $OA$ and $OF$ take $OP$ and $OQ$ equal to $OZ$ or $OM$.

    Then $PM$, $MZ$, and $ZQ$ are three sides of the pentedecagon.

    Treating similarly the angles $AOB$, $BOC$, $COD$, and $DOZ$, we obtain the
    remaining sides of the pentedecagon.

\end{enumerate}


%%%%%%%%%%%%%%%%%%%%%%%%%%%%%%%%%%%%%%%%%%%%%%%%%%%%%%%%%%%%%%%%%%%%%%%%%%%%%%%%
