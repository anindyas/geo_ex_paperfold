%%%%%%%%%%%%%%%%%%%%%%%%%%%%%%%%%%%%%%%%%%%%%%%%%%%%%%%%%%%%%%%%%%%%%%%%%%%%%%%%

\chapter{POLYGONS.}


\begin{enumerate}

\item Find $O$, the center of a square by folding its diameters.  Bisect the right
      angles at the center, then the half right angles, and so on.  Then we
      obtain $2"$ equal angles around the center, and the magnitude of each of the 
      angles is $=$ of a right angle, $m$ being a positive integer.  Mark off equal 
      lengths on each of the lines which radiate from the center.  If the 
      extremities of the radii are joined successively, we get regular polygons 
      of $2”$ sides.


%\item Let us find the perimeters and areas of these polygons. In Fig. 47 let OA
        %and OA; be two radii at right angles to each other. Let the radii OAp,
        %OA3, OAg4, etc., divide the right angle 4,04 into 2, 4, 8....parts.
        %Draw 4A\, AA, AA3....cutting the radii OA», OA3, OAy....at Bi, By,
        %B3....respectively, at right angles. Then 4), By, B3....are the
        %mid-points of the respective chords.
%Then 44), AAp, AA3, AAy....are the sides of the inscribed polygons of 2?, 23,
        %24....sides respectively, and OF), OB2....  are the respective
        %apothems.  LewoA = 2,
%
%
%a(2") represent the side of the inscribed polygon
%
%of 2" sides, 6(2") the corresponding apothem, p(2”) its perimeter, and A(2”)
        %its area.  For the square, a(2?) — RY 2: (2) =R2?-V2;
%
%
%b(22) = 5V2;
%
%
%A (2%) = R?2,
%
%For the octagon, in the two triangles 42,0 and AB\A2 AB, OA
%
%
%%ByA, AA +. $4 Ag? = R+ Bi Ay = RLR—0(29)]
%
%Ym Re ae AE
%
%
%R224 V2 = eo fe DER 2 EVE (8)
%
%A(2*) =} perimeter X apothem
%
%
%= R-22-1/2—VY2 ARY24V2 — 2-22.  Similarly for the polygon of 16 sides,
%
%
%a()= RY 2—V2+V2; PCA)HR-2/2—-Y 2+ V2; b= Y/ 24/24 723
%
%
%A (24) == R?-2?V2—V2; pag os and for the polygon of 32 sides,
%
%
%a(28)—=R/ 21/24 V 242; panne Vaan V env = FV 24/24 VOTES A (25) = R222 1/242.
%
%The general law is thus clear.
%
%
%Also A(2";== S -p(2"-1).
%
%
%As the number of sides is increased indefinitely
%
%the apothem evidently approaches its limit, the radius. Thus the limit of
%
%
%/2+V2+V2....is 2; for if x represent the limit, «—W2-+ x, a quadratic which
        %gives x2, or —1; the latter value is, of course, inadmissible.
%
%
\item If perpendiculars are drawn to the radii at their extremities, we get
    regular polygons circumscribing the circle and also the polygons described
    as in the preceding article, and of the same number of sides.

        %.  < F G
%
%°
%
%
%Fig. 48.
%
%
%In Fig. 48, let 4 be a side of the inscribed polygon and FG a side of the
        %circumscribed polygon.  Then from the triangles F7Z and £/0O, OZ FE FG.
        %OL. Bil An AE
%
%
%) FG= Ra
%
%The values of AZ and O/ being known by the previous article, ¥G is found by
        %substitution.
%
%The areas of the two polygons are to one another as #G?:A#*, i. e., as R*: O17;
%
%\item In the preceding articles it has been shown how regular polygons can be
        %obtained of 2?, 28....2” sides. And if a polygon of m sides be given,
        %it is easy to obtain polygons of 2*-m sides.
%
%
%\item In Fig. 48, 42 and CD are respectively the sides of the inscribed and
        %circumscribed polygons of asides. Take £ the mid-point of CD and draw
        %AZ, BE.  A&and BE are the sides of the inscribed polygon of 2x sides.
%
%Fold 4/7, 2G at right angles to dC and BD, meeting CD in Fand G.
%
%Then FG isa side of the circumscribed polygon of 2x sides.
%
%Draw OF, OG and OZ.
%
%Let 4, P be the perimeters of the inscribed and circumscribed polygons
        %respectively of # sides, and
%
%_A, # their areas, and /’, P’ the perimeters of the inscribed and circumscribed
        %polygons respectively of 2” sides, and 4’, #’ their areas.
%
%Then
%
%Pua 4S, P=2-CD, pf =—2eAL, P= 2nFG.
%
%Because OF bisects 7 COZ, and AB is parallel
%
%
%tO1C2).  CAE Fao co CL
%
%
%FE~ OE ~ AO” AB’
%
%CE (CD+4 AB
%
%ro oie? mad is. = 4n-CE _n-CD+n:AB
%
%4n-FE nAB 2P.,P+2 eye eae
%
%> 2Lp Again, from the similar triangles E/¥ and 4HE,
%
%El” Bee AH AE.
%
%
%or AM =2AH TES; . 42 -AkA=4n)-AB- EF, or ges Pip.  Now, A=2nnAOH, B=2na COE#,
        %A'’=2nKR AOE, B'=4An KR LFOE.  The triangles 4OH and AOZ are of the
        %same altitude, 4H, AAOH _OH
%
%
%* RAOE ~ OF
%
%Similarly, AAOE _OA ECOER 2.00.
%
%
%Again because AB || CD, , A007. BAOE “AAOE” ACOE A A! — Ss — / s peer A’ =V AB
%
%
%Now to find B’. Because the triangles COZ# and
%
%FOE have the same altitude, and OF bisects the angle EOC, ASCOE CK OC+O0OE ALOR
        %FRE Of, and OE = OA, OC GOLIATH.  O62 OT! KADY.  ACOE AAOE+ AAOH
%
%and
%
%bahia AAOH From this equation we easily obtain ed =: os se it B Da yee
%
%Fig. 49.
%
%
%\item Given the radius & and apothem of a regular polygon, to find the radius
        %#’ and apothem 7’ of aregular polygon of the same perimeter but of
        %double the number of sides.
%
%Let 4B be a side of the first polygon, O its center, OA the radius of the
        %circumscribed circle, and OD the apothem. On OD produced take OC=OA or
        %OB. Draw AC, BC. Fold OA’ and OBS’ perpendicular to AC and BC
        %respectively, thus fixing the points 4’, B’. Draw A'B’ cutting OC in
        %D’. Then the chord 4’S’ is ha'f of 4B, and the angle B’O4’
%is half of BOA. OA' and OD' are respectively the radius 2’ and apothem 7’ of
        %the second polygon.
%
%Now OD’ is the arithmetic mean between OC and OD, and OA’ is the mean
        %proportional between OC and OD".
%
%
%’ ar gee es
%
%
%4(R+ vn, and P=V Rr.
%
%
%\item Now, take on OC, OEF=OA' and draw /4’E£.
%
%Then 4'D’ being less than 4’C, and / D’A'C being bisected by 4’Z,
%
%ED is less than CD’, i. e., less than } CD *. R,—r is less than }(R—r).
%
%As the number of sides is increased, the polygon approaches the circle of the
        %same perimeter, and & and ~ approach the radius of the circle.
%
%That is,
%
%
%R+r+t+Ry—nt+ Ro—re+.. ook
%
%
%= the diameter of the circle = f Also, RY = Rr or R: z =R, and E =F and so on.
        %Multiplying both sides, ° z . %. . ze ..== the radius of the circle =f.
%
%\item The radius of the circle lies between #,, and
%
%
%r,, the sides of the polygon being 4:2* in number;
%
%
%and z lies between ez and ee The numerical value of z can therefore be
        %calculated to any required degree of accuracy by taking a sufficiently
        %large number of sides.
%
%The following are the values of the radii and apothems of the regular polygons
        %of 4, 8, 16....2048 sides.
%
%4-gon, r=0°500000 R=rV/2=0-707107 8-gon, 71=0-603553 A&R; —0-653281 2048-gon, 7
        %—0-636620 Ry—0-636620.  2
%
%
%a8 = 97956600 => 14159----
%
%
%\item If RX” be the radius of a regular isoperimetric
%
%
%polygon of 47 sides
%
%aR ee) 2R ‘ or in general R 1 & Ress = f Fy} A ire 2 140. The radii #i,
        %R,....successively diminish,
%
%
%and the ratio sis less than unity and equal to the 1
%
%
%cosine of a certain angle a.
%
%
%Ae [k COs a Fs xP FORE coe S
%
%Rea ee a multiplying together the different ratios, we get
%
%
%a a Rix, =F, cosa-cos > + cos5,... COS 5G
%
%
%2
%
%
%when =o,
%
%Te a a The limit of cosa@-cos=,....cos ; of 92 rad
%
%sin2a@ :
%
%5 a result known as Luder’s Formula.  2a
%
%\item It was demonstrated by Karl Friedrich Gauss* (1777-1855) that besides the
        %regular polygons of 2”, 3-2", 5-2", 15-2" sides, the only regular
        %polygons which can be constructed by elementary geometry are those the
        %number of whose sides is represented by the product of 2" and one or
        %more different numbers of the form 2"-+1. We shall show here how
        %polygons of
%5 and 17 sides can be described.
%
%The following theorems are required :t
%
%(1) If Cand D are two points on a semi-circumference ACDB, and if C’ be
        %symmetric to C with respect to the diameter 42, and # the radius of the
        %circle,
%
%
%AC-BD=R:+(C'D—CD)....... i, AD:+-BC=R:+(C'D+CD)...... ii.  AC BG BCL. ees iii,
%
%
%(2) Let the circumference of a circle be divided into an odd number of equal
        %parts, and let 40 be the diameter through one of the points of section
        %4 and the mid-point O of the opposite arc. Let the points of section on
        %each side of the diameter be named 4}, Ao, As....A,, and A’;, A’o,
        %A’3....A’, beginning next to A.  ibien OAt OAs OAs.2- 4 OA, — Kee iv.
%and O4;:OA2-OA....OA,= 2.
%
%
%*Beman and Smith’s translation of Fink's History of Mathematics, p.  245; see
        %also their translation of Klein's Famous Problems of Elementary
        %Geometry, pp. 16, 24, and their New Plane and Solid Geometry, p, 212.
%
%
%+These theorems may be found demonstrated in Catalan’s 7éordmes et Problémes de
        %Géométrie Elémentaire,
%
%
%
%\item It is evident that if the chord OA, is determined, the angle 4,,OA is
        %found and it has only to be divided into 2” equal parts, to obtain the
        %other chords.
%
%Fig. 50
%
%
%\item Let us first take the pentagon.  By theorem iv, OA,: OAq= R?.  By theorem
        %i,
%
%R(0A,— 042)= 0A, OAo= RF.
%
%
%78 GEOMETRIC EXERCISES a Od ps My ee R .  *, OA=Z (Y5+1),
%
%
%and OA, =5 (175 —1).
%
%
%Hence the following construction.
%
%Take the diameter ACO, and draw the tangent AF, Take Dthe mid-point of the
        %radius OC and Am UC,
%
%On OC as diameter describe the circle AZ'CZ.
%
%Join /D cutting the inner circle in # and Z".
%
%Then “Z' = OA, and FE= OA.
%
%
%\item Let us now consider the polygon of seventeen sides.  Here* OA: OA: OA3:
        %OA4g* OA5* OAG* OA7* OAg= RS.  OA\: OA2: OAg: OAg=R4.  and
        %OA3:OA;~OA¢:OA;=R'.  By theorems i. and ii.  OA,: OA,= R(OA3+4 OAs) OA:
        %OAxg = R (OAg— OA?)  OA3* OAs = R(OA2+ OAs) OA,¢: OA;= R(OA\— OA4)
        %Suppose OA3+ OAs=M, OAg— OA1=N, OA,+ OAg=P, OA\—OA4=0.  *The principal
%steps are given. For a full exposition see Catalan's Théorémes et Problémes de
        %Glométrie Elémentaire, The treatment is given in full
%
%
%in Beman and Smith's translation of Klein's Famous Problems of Elementary
        %Geometry, chap. iv.
%
%Then MNV= FF? and POS k?, Again by substituting the values of 4%, WV, P and Q
        %in the formulas MN=—R?, PO=R?  and applying theorems i. and ii. we get
        %(M—N)—(P—Q)=R Also by substituting the values of 17, VV, P and Qin
%
%
%the above formula and applying theorems i. and ii.
%
%
%we get (M— NV) (P— Q) = 4R?.
%
%
%Hence ”—WN, P—Q, M, NV, P and Q are determined.
%
%
%Again 04:4 O47,
%
%
%OA: OAxs= RN.
%
%
%Hence O4As is determined.
%
%
%\item By solving the equations we get M—N=}R(14+V 17).  P—Q=}k(—1+4+V 1%.
%
%= eR 17-4 | — 2717).  N=4R(—1—V 1741344 2717).  OA, =3R[—14 V1IT4+// 34—2//17
%
%
%—2)/17 438 17+ /110—26 / 174/344-2717 | =4Rk[—1+4+V174+/ 34—2V17
%
%
%—8i/ 17-431 Tt — 1/170 + 38 17 |.
%
%\item The geometric construction is as follows:
%
%Let BA be the diameter of the given circle; O its center. Bisect OA in C. Draw
        %4D at right angles to OA and take AD=AB. DrawCD. Take £ and Z’ in CD
        %and on each side of C so that CE= CZ’ =CA.
%
%Fig. 51.
%
%
%Bisect ZD in G and £’D in G’. Draw DF perpendicular to CD and take D¥= OA.
        %Draw /G and 7G’.  Take # in F/G and #’ in FG’ produced so that GH= EG
        %and G'H’=G'D.  Then it is evident that DE=M—WN, DE' =P—Q;
%
%also Pile Po (FA DE) FAS Dra.
%
%Again in DF take X such that FK=/FH.
%
%Draw XZ perpendicular to D/'and take Z in KZ such that /Z is perpendicular to
        %DZ.
%
%hea Ab =. 0h* 7K =F.
%
%Again draw HN perpendicular to #H’ and take H'N=FL. Draw VM perpendicular to
        %VH’. Find M in NM such that H’M is perpendicular to “7M.  Draw /F’
        %perpendicular to “H’.
%
%Then
%
%OID (oma MPa FIP = RN.
%
%But #/’+ /H' =P.  me dt CAG.
%
\end{enumerate}

%%%%%%%%%%%%%%%%%%%%%%%%%%%%%%%%%%%%%%%%%%%%%%%%%%%%%%%%%%%%%%%%%%%%%%%%%%%%%%%%
