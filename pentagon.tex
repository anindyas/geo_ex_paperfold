%%%%%%%%%%%%%%%%%%%%%%%%%%%%%%%%%%%%%%%%%%%%%%%%%%%%%%%%%%%%%%%%%%%%%%%%%%%%%%%%

\chapter{THE PENTAGON.}

\begin{enumerate}

    \item To cut off a regular pentagon from the square ABCD. 7

        Divide $BA$ in $X$ in median section and take $YW$ the mid-point of
        $4X$.

%Fig. 27.
%
%Then 4B: AX=XB?, and AM= MX.
%
%Take BVN=AM or MX.
%
%Then MV= XB.
%
%Lay off VP and WR equal to AZN, so that P and R may lie on BC and AD
%respectively.
%
%Lay off RQ and PO=WMR and WP.
%
%MNPOQR is the pentagon required.
%
%In Fig. 19, p. 22, 4, which is equal to 4B, has the point JV on the
%perpendicular 7O. If A be moved on AB over the distance 7A, then it is
%evident that JV will be moved on to BC, and X to
%
%Therefore, in Fig. 27, VR=AB. Similarly “P= AB. RP is also equal to AZ
%and parallel to it.
%
%LRMA =} of a tt. /.  we dV A ee Ola ttf
%
%Similarly LENM =F ofa tt. 2. .
%
%From triangles VR and QORP, 7 NMR= 7 ROP == of a rt. 7.
%
%The three angles at 17, 4, and Q of the pentagon being each equal to § of
%a right angle, the remaining two angles are together equal to 12 of a right
%angle, and they are equal. Therefore each of them is £ of a right angle.
%
%Therefore all the angles of the pentagon are equal.
%
%The pentagon is also equilateral by construction.
%
%was ae base AZZ of the pentagon is equal to XB, og ees Bay
%5—1)=ABX0.6180.... § 58.  The sie test breadth of the pentagon is 4B.
%
%
%\item If p be the altitude, 2 AB = pr Es vWe— 1)| 51/5.  eee Sa —V 5.
%-paap.(1—8 | — pe PTV S 8 V 2V5 Sn af eerie dee se a == ABO 2510. cane A Cos 18,
%
%AY
%
%
%Fig. 28.
%
%\item If & be the radius of the circumscribed circle,
%
%‘AB eee ~ 2cos18° 104275 5— V5.  aera
%
%
%— AB 0.5257...
%
%\item If r be the radius of the inscribed circle, then from Fig. 28 it is
%evident that
%
%
%Pops Re ABs cote — AB: pice =AB+V/b4+V5. (J 3 +425") =4n-Voqve [oe
%
%
%Vv 40 Legs ses
%
%
%=AB 0.4203 556 76. The area of the pentagon is 57 x } the base of the
%pentagon, i. e.,
%
%
%[545 | au
%
%
%Bea -(V5—1)
%
%
%5 22 =A —jo = 4B X 0.6571...
%
%
%\item In Fig. 27 let PR be divided by WQ and VQ in £ and /.
%
%
%%Then MN= 22 (5-1). $72
%
%
%Pee ee ee eee pe ot 2 cos 36° SSE =o eet a
%
%EF=AB—2RE=AB—AB(3—V 5) =AB(V5—2)... (2)
%
%
%RF=MN.  RF: RE == RE» EF (by § 51)... 0.52202 eens en cease Y6—173 V6
%=3—Y 5: 2(V 6 —2).. ie venene (4) By § 76 the area of the pentagon 5
%[5—V5 VeHiv 8 obeys Sas 7. pi tose C8 pe A =a -( 2 4 10 = MN: 1/25 +
%10V5, since AB= un yes
%
%
%.'. the area of the inner pentagon = EF?4 : oes 10V5 = AB (Y5—2) +
%5 1 1/2 +105.
%
%
%The larger pentagon divided . the smaller = MN? : EF?  =2:(7T—3Vv 5)
%=1 : 0.145898...
%
%\item If in Fig. 27, angles QEX and ZFOQ are made equal to 2RQ or FOP, K, Z
%being points on the sides QR and QP respectively, then EFL QK will be a regular
%pentagon congruent to the inner pentagon. Pentagons can be similarly described
%on the remaining sides of the inner pentagon. The resulting figure consisting of
%six pentagons is very interesting.
%
\end{enumerate}

%%%%%%%%%%%%%%%%%%%%%%%%%%%%%%%%%%%%%%%%%%%%%%%%%%%%%%%%%%%%%%%%%%%%%%%%%%%%%%%%
