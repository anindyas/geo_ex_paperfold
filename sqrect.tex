%%%%%%%%%%%%%%%%%%%%%%%%%%%%%%%%%%%%%%%%%%%%%%%%%%%%%%%%%%%%%%%%%%%%%%%%%%%%%%%%

\chapter{SQUARES AND RECTANGLES.}


\begin{enumerate}

    \item Fold the given square as in Fig. 13. This affords the well-known proof
        of the Pythagorean theorem.  $FGH$ being a right-angled triangle, the
        square on $FH$ equals the sum of the squares on $FG$ and $GH$.

        $$\square FA + \square DB = \square FC$$.

        It is easily proved that $FC$ is a square, and that the triangles $FGH$,
        $HBC$, $KDC$, and $FEK$ are congruent.

        If the triangles $FGH$ and $HBC$ are cut off from the squares $FA$ and
        $DB$, and placed upon the other two triangles, the square $FHCK$ is made
        up.

%If 4B=2, GA=d, and FH=c, then.2 += 2’.

    \item Fold the given square as in Fig. 14. Here the rectangles 47, $BG$,
        $CH$, and $DE$ are congruent, as also the triangles of which they are
        composed.  $E7XGH$ is a square as also $KZMN$.

%et 4k ——2,14h eb =>, and Vk =<,
%
%then a?+ 2?=—¢?, i.e. 0OKLMN.
%
%oABCD=(a- 4).  Now square ABCD overlaps the square KLMN by the four triangles
%AK, BLK, CML, and DNM.  But these four triangles are together equal to
%two of the rectangles, i.  e., to 2aé.  Therefore (a+ 0)? =a? + 6? + 2ad.
%
%
%\item EF=—a—dé, ando EFGH= (a—)’.  The square £¥G/H is less
%than the square AL MN by the four triangles PVK, GAL, HLM, and EMN.
%But these four triangles make up two of the rectangles, i. e., 2ad.
%.t. (a—b)? =a? + 6? — 2adb.
%
%Fig. 15 Fig. 16.  Fig. 17.
%
%
%\item The square ABCD overlaps the square E¥GH by the four rectangles 4/7,
%BG, CH, and DE.  .. (a+ 4)? — (a—b)? =4ab.
%
%\item In Fig. 15, the square 4BCD=(a- 6)’, and the square
%-F¥GH=(a—d)*. Also square AKGN = square EZLCM=a*. Square KBLF = square
%NHMD = 0’.
%
%Squares ABCD and EFGH are together equal to the latter four squares put
%together, or to twice the square d4XGW and twice the square KBLF, that is,
%
%(a+ 4)? + (a— 4)? = 2a? + 207.
%
%
%\item In Fig. 16 the rectangle PZ is equal to (a+ 4) (@—2).
%
%Because the rectangle EX — FM, therefore rectangle PZ = square PAX
%—square A#, i. e., (a+) (a— 6) =a? — 3b.

    \item If squares be described about the diagonal of a given square, the
        right angle at one corner being common to them, the lines which join
        this corner with the mid-points of the opposite sides of the given
        square bisect the corresponding sides of all the inner squares. (Fig.
        17.) For the angles which these lines make with the diagonal are equal,
        and their magnitude is constant for all squares, as may be seen by
        superposition. Therefore the mid-points of the sides of the inner
        squares must lie on these lines.

    \item $ABCD$ being the given square piece of paper (Fig. 18), it is required
        to obtain by folding, the point $X$ in $AB$, such that the rectangle
        $AB-XVP$ is equal to the square on $AX$.

        Double $AC$ upon itself and take its mid-point $Z$.

        Fold through $Z$ and $4$.

%Lay £B upon £A and fold so as to get £F, and G such that EG=£B.
%
%Take AX= AG.
%
%Then rectangle 4B: XB=AX’”.
%
%Complete the rectangle BCHX and the square AXKL.
%
%Let XH cut HA in MW. Take FY=/FB.
%
%Then FB =/G=FY=XM and XM=4AX,
%
%Fig. 18.
%
%Now, because #Y is bisected in # and produced to A, AB:AY+ FY°=AF”,
%by § 49, =AG?+ FG", by § 44.  siete AY a A, oe ks But AX* 224° NA on B 4,
%
%24 AX DY and Ay AX, sn ABN BAN es AB is said to be divided in X in median
%section. * Also AB-AY=BY%*, i. e., 4B is also divided in Y in median section.
%
    
    \item A circle can be described with $F$ as a center, its circumference
        passing through $A$, $G$, and Y.  It will touch $ZA$ at $G$, because 
        $/G$ is the shortest distance from F to the line $EGA$.

%\item Since BH=BN, subtracting BX we have rectangle XX/VY = square CHP,
%i. e., AX: YX=AY?,
%
%i. e., AX is divided in Y in median section.  Similarly AY is divided in
%X in median section.  54. °.: 4B: XB=AX?
%
%.. 84B:-XB=AX?+ BX:-BC+ CD-CP — ABI + BX?,
%
%\item Rectangles BH and YD being each=AB- XB, rectangle HY + square CK =
%AX? = AB: XB.
%
%\item Hence rectangle 7Y= rectangle BK, i. e., AX: XB=AB:XY.
%
%\item Hence rectangle HV=AX:-XB— BX?.
%
%*The term “ golden section” is also used. See Beman and Smith's Vew
%Plane and Solid Geometry, p. 196.
%
%\item Let A4B=a, XB =x.  Then (a— x)? =ax, by § 51.  a? + x? —3ax,
%by § 54;
%
%and =F (3—V5).  ao : esta nae: =z (7—3V'5).
%
%
%a x= 5 (VbB—1) =a X 0.6180...
%
%
%2 = +. (@—x) =F (83— V5) =a? x 0.3819...  The rect. BPAX = (a—x)x
%=a? (V5—2)=a’x0.2360....  EA'=5sEB? — 3 AB BA = 4B =1.1180. eo.
%
%
%\item In the language of proportion AD AX aA Kae XB The straight line 4B is
%said to be divided ‘in extreme and mean ratio.”

    \item Let $AZ$ be divided in $X$ in median section.  Complete the rectangle
        $CBXH$ (Fig.  19).  Bisect the rectangle by the line $A/VO$. Find the
        point $WV$ by laying $XA$ over $X$ so that $A$ falls on $AZO$, and fold
        through $XV$, $VB$, and $VA$.  Then $BAW$ is an isosceles triangle
        having its angles $AB$ and $BNA$ double the angle $VAB$.

%AX=AN=NLB LABN=ZNXB LINNAX== SANA LNXB=2/NAX
%
%Fig. 19.
%
%
%L ABN=2/ NAB.  AN*=MN?4+ AM?
%
%— BN*—BM?+ AM?
%
%—AX2+4 AB-AX
%
%—AB-XB1+AB:AX
%
%— AB?
%
%<i AVA, and LNAB =} of a right angle.
%
%
%\item The right angle at 4 can be divided into five equal parts as in Fig. 20.
%Here WV’ is found as in § 60. Then fold AW'Q; bisect / QAB by folding,
%fold over the diagonal AC and thus get the point ks
%
%A B
%
%
%Fig. 20.
%
%\item To describe a right-angled triangle, given the hypotenuse 4BZ, and
%the altitude.
%
%Fold £F (Fig. 21) paraHel to AB at the distance of the given altitude.
%
%Take G the middle point of 42. Find # by folding GB through G so that B
%may fall on 2/.
%
%Fold through H and 4, G, and B.  AB is the triangle required.
%
%Fig. 2r.
%

    \item $ABCD$ (Fig: 22) is a rectangle.  It is required to find a square
        equal to it in area.

%Mark off B= BC.  Find O, the middle point of 4J/, by folding.

%Fold OM, keeping O fixed and letting J/ fall on line BC, thus finding /,
%the vertex of the right-angled triangle 41/P.
%
        Describe on $PB$ the square $BPQOR$.

%The square is equal to the given rectangle. |
%
%For -.3P=QP, and the angles are equal, triangle BMP is evidently congruent
%to triangle QSP.
%
%eto == DM == AD.
%
%.*. triangles DA 7 and QSP are congruent.
%
%... PC=SR and triangles RSA and CPT are congruent.
%
%.*. [ABCD can be cut into three parts which can be fitted together to form
%the square RAPQ.
%
%Fig. 23 Fig. 24.  Fig. 25.

    \item Take four equal squares and cut each of them into two pieces through
        the middle point of one of the sides and an opposite corner. Take also
        another equal square. The eight pieces can be arranged round the square
        so as to form a complete square, as in Fig.  23, the arrangement being a
        very interesting puzzle.  The fifth square may evidently be cut like the
        others, thus complicating the puzzle.  

    \item Similar puzzles can be made by cutting the squares through one corner
        and the trisection points of the opposite side, as in Fig.  24.  areas

    \item If the nearer point is taken 10 squares are required, as in Fig. 24;
        if the remoter point is taken 13 squares are required, as in Fig. 25.  

    \item The puzzles mentioned in §§ 65, 66, are based upon the formulas

%12+ 2?—=5 124+ 3710 22.4 3? — 13.

        The process may be continued, but the number of squares will become
        inconveniently large.

    \item Consider again Fig. 13 in § 44. If the four triangles at the corners
        of the given square are removed, one square is left. If the two
        rectangles $XK$ and $XG$ are removed, two squares in juxtaposition are
        left.

    \item The given square may be cut into pieces which can be arranged into two
        squares.  There are various ways of doing this. Fig. 23, in § 65,
        suggests the following elegant method: The required pieces are 

        \begin{enumerate}[(1)]
            \item the square in the center, and 
            \item the four congruent symmetric quadrilaterals at the corners,
                together with the four triangles. 
        \end{enumerate}
    
        In this figure the lines from the mid-points of the sides pass through
        the corners of the given square, and the central square is one-fifth of
        it. The magnitude of the inner square can be varied by taking other
        points on the sides instead of the corners.

    \item The given square can be divided as follows (Fig. 26) into three equal
        squares: Take $BG = \text{half the diagonal of the square}$.

%Fig. 26.

    Fold through C and G.

    Fold BM perpendicular to CG.

    Take MP, CN, and VZ each = BM.
    
    Fold PH, NX, ZF at right angles to CG, as in Fig. 26.
    
    Take VK = BM, and fold K£ at right angles to NE.
    
    Then the pieces 1, 4, and 6, 3 and 5, and 2 and 7 form three equal squares.
    
    Now CG?=3BG’, and from the triangles GBC and CVB BM BG.  Pieris
    
    Letting BC—=a, we have
    
\end{enumerate}

%%%%%%%%%%%%%%%%%%%%%%%%%%%%%%%%%%%%%%%%%%%%%%%%%%%%%%%%%%%%%%%%%%%%%%%%%%%%%%%%
