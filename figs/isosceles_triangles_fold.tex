% Nice shaded/framed paragraphs using tikz and framed
% Author: Jose Luis Diaz

\begin{tikzpicture}

    \coordinate (O) at (0,0);
    \coordinate (A) at (-3,-1);
    \coordinate (B) at ( 3,-1);
    \coordinate (C) at (-1.5,4);

    \draw (A) -- (B) -- (C) -- (A);
    \foreach \x in {A, B, C}
        \draw (O) -- (\x);
    
    \draw (O) -- ($(A)!(O)!(B)$);

    \node[label={below:$A$}] at (A) {};
    \node[label={below:$B$}] at (B) {};
    \node[label={above:$C$}] at (C) {};

    \node[label={north west:$Z$}] at ($(A)!(O)!(B)$) {};

\end{tikzpicture}


%%%%%%%%%%%%%%%%%%%%%%%%%%%%%%%%%%%%%%%%%%%%%%%%%%%%%%%

% define styles for the normal border and the torn border
\tikzset{
  normal border/.style={orange!30!black!10, decorate,
     decoration={random steps, segment length=2.5cm, amplitude=.7mm}},
  torn border/.style={orange!30!black!5, decorate,
     decoration={random steps, segment length=.5cm, amplitude=1.7mm}}}

% Macro to draw the shape behind the text, when it fits completly in the
% page
\def\parchmentframe#1{
\tikz{
  \node[inner sep=2em] (A) {#1};  % Draw the text of the node
  \begin{pgfonlayer}{background}  % Draw the shape behind
  \fill[normal border]
        (A.south east) -- (A.south west) --
        (A.north west) -- (A.north east) -- cycle;
  \end{pgfonlayer}}}


% Define the environment which puts the frame
% In this case, the environment also accepts an argument with an optional
% title (which defaults to ``Example'', which is typeset in a box overlaid
% on the top border
\newenvironment{parchment}[1][Example]{%
  \def\FrameCommand{\parchmentframe}%
  \def\FirstFrameCommand{\parchmentframetop}%
  \def\LastFrameCommand{\parchmentframebottom}%
  \def\MidFrameCommand{\parchmentframemiddle}%
  \vskip\baselineskip
  \MakeFramed {\FrameRestore}
  \noindent\tikz\node[inner sep=1ex, draw=black!20,fill=white,
          anchor=west, overlay] at (0em, 2em) {\sffamily#1};\par}%
{\endMakeFramed}

%%%%%%%%%%%%%%%%%%%%%%%%%%%%%%%%%%%%%%%%%%%%%%%%%%%%%%%

\newcounter{mathseed}
\setcounter{mathseed}{3}
\pgfmathsetseed{\arabic{mathseed}} % To have predictable results
% Define a background layer, in which the parchment shape is drawn
\pgfdeclarelayer{background}
\pgfsetlayers{background,main}

% This is the base for the fractal decoration. It takes a random point between
% the start and end, and raises it a random amount, thus transforming a segment
% into two, connected at that raised point This decoration can be applied again
% to each one of the resulting segments and so on, in a similar way of a Koch
% snowflake.
\pgfdeclaredecoration{irregular fractal line}{init}
{
  \state{init}[width=\pgfdecoratedinputsegmentremainingdistance]
  {
    \pgfpathlineto{\pgfpoint{random*\pgfdecoratedinputsegmentremainingdistance}{(random*\pgfdecorationsegmentamplitude-0.02)*\pgfdecoratedinputsegmentremainingdistance}}
    \pgfpathlineto{\pgfpoint{\pgfdecoratedinputsegmentremainingdistance}{0pt}}
  }
}


% define some styles
\tikzset{
   
    paper/.style={draw=black!10, blur shadow, 
                  every shadow/.style={opacity=1, black}, 
                  shade=bilinear interpolation, 
                  lower left=black!10, upper left=black!5, 
                  upper right=white, lower right=black!5, 
                  fill=none},

    irregular cloudy border/.style={decoration={irregular fractal line, 
                                    amplitude=0.2}, decorate,},

    irregular spiky border/.style={decoration={irregular fractal line, 
                                   amplitude=-0.2}, decorate,},

    ragged border/.style= {decoration={random steps, segment length=7mm, 
                                       amplitude=2mm}, decorate,}
}

\def\tornpaper#1{

    \ifthenelse{\isodd{\value{mathseed}}}{%
    \tikz{
      \node[inner sep=1em] (A) {#1};  % Draw the text of the node

      \begin{pgfonlayer}{background}  % Draw the shape behind

      \fill[paper] % recursively decorate the bottom border
         \pgfextra{\pgfmathsetseed{\arabic{mathseed}}\addtocounter{mathseed}{1}}
          {decorate[irregular cloudy border]{decorate{decorate{decorate{decorate[ragged border]{
            (A.north west) -- (A.north east)
          }}}}}}
          -- (A.south east)
         \pgfextra{\pgfmathsetseed{\arabic{mathseed}}}%
          {decorate[irregular spiky border]{decorate{decorate{decorate{decorate[ragged border]{
          -- (A.south west)
          }}}}}}
          -- (A.north west);

      \end{pgfonlayer}}
    }

    {
    \tikz{
      \node[inner sep=1em] (A) {#1};  % Draw the text of the node

      \begin{pgfonlayer}{background}  % Draw the shape behind

      \fill[paper] % recursively decorate the bottom border
         \pgfextra{\pgfmathsetseed{\arabic{mathseed}}\addtocounter{mathseed}{1}}%
          {decorate[irregular spiky border]{decorate{decorate{decorate{decorate[ragged border]{
            (A.north east) -- (A.north west)
          }}}}}}
          -- (A.south west)
         \pgfextra{\pgfmathsetseed{\arabic{mathseed}}}%
          {decorate[irregular cloudy border]{decorate{decorate{decorate{decorate[ragged border]{
          -- (A.south east)
          }}}}}}
          -- (A.north east);

      \end{pgfonlayer}}
    }
}

%\tornpaper{ \parbox{.9\textwidth}{\lipsum[11]} }

%\begin{tikzpicture}
%
%    \begin{axis}[colormap/blackwhite, view={30}{60}, axis lines=none]
%
%       \addplot3[surf,shader=interp, samples=60, point meta=x+3*z*z-0.25*y,
%                 domain=0:2*pi,y domain=0:2*pi, z buffer=sort]
%       ({(2+cos(deg(x)))*cos(deg(y))},
%        {(2+cos(deg(x)))*sin(deg(y))},
%        {sin(deg(x))});
%
%   \end{axis}
%
%\end{tikzpicture}
