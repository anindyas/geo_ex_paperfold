
%%%%%%%%%%%%%%%%%%%%%%%%%%%%%%%%%%%%%%%%%%%%%%%%%%%%%%%%%%%%%%%%%%%%%%%%%%%%%%%%

\chapter{THE SQUARE.}

\begin{enumerate}

    \item The upper side of a piece of paper lying flat upon a table is a plane
        surface, and so is the lower side which is in contact with the table.

    \item The two surfaces are separated by the material of the paper. The
        material being very thin, the other sides of the paper do not present
        appreciably broad surfaces, and the edges of the paper are practically
        lines. The two surfaces though distinct are inseparable from each other.

    \item Look at the irregularly shaped piece of paper shown in Fig. 3, and at
        this page which is rectangular. Let us try and shape the former paper
        like the latter.

    \item Place the irregularly shaped piece of paper upon the table, and fold
        it flat upon itself. Let $X'X$ be the crease thus formed. It is
        straight. Now pass a knife along the fold und separate the smaller
        piece.  We thus obtain one straight edge.
    
    \item Fold the paper again as before along $BY$, so that the edge $X'X$ is
        doubled upon itself. Unfolding the paper, we see that the crease $BY$ is
        at right angles to the edge $X'X$. It is evident by superposition that
        the angle $YBX'$ equals the angle $XBY$, and that each of these angles
        equals an angle of the page. Now pass a knife as before along the second
        fold and remove the smaller piece.
    
    \item Repeat the above process and obtain the edges $CD$ and $DA$. It is
        evident by superposition that the angles at $A$, $B$, $C$, $D$, are
        right angles, equal to one another, and that the sides $BC$, $CD$ are
        respectively equal to $DA$, $AB$.  This piece of paper (Fig. 3) is
        similar in shape to the page.
    
    \item It can be made equal in size to the page by taking a larger piece of
        paper and measuring off $AB$ and $BC$ equal to the sides of the latter.
    
    \item A figure like this is called a rectangle.  By superposition it is
        proved that 

        \begin{enumerate}[(1)]
            \item the four angles are right angles and all equal, 
            \item the four sides are not all equal, 
            \item but the two long sides are equal, and so also are the two
                short sides.
        \end{enumerate}

    \item Now take a rectangular piece of paper, $A'B’CD$, and fold it obliquely
        so that one of the short sides, $CD$, falls upon one of the longer
        sides, $DA'$, as in Fig. 4.  Then fold and remove the portion $A'B'BA$
        which overlaps. Unfolding the sheet, we find that $ABCD$ is now square,
        i.e., its four angles are right angles, and all its sides are equal.
    
    
    \item The crease which passes through a pair of the opposite corners $B$,
        $D$, is a diagonal of the square.  One other diagonal is obtained by
        folding the square through the other pair of corners as in Fig. 5.
    
    \item We see that the diagonals are at right angles to each other, and that
        they bisect each other.
    
    \item The point of intersection of the diagonals is called the center of the
        square.
    
    \item Each diagonal divides the square into two congruent right-angled
        isosceles triangles, whose vertices are at opposite corners.
    
    \item The two diagonals together divide the square into four congruent
        right-angled isosceles triangles, whose vertices are at the center of
        the square.
    
    \item \label{item:square} Now fold again, as in Fig. 6, laying one side cf
        the square upon its opposite side. We get a crease which passes through
        the center of the square. It is at right angles to the other sides and 

    \begin{enumerate}[(1)]
        \item bisects them; 
        \item it is also parallel to the first two sides; 
        \item it is itself bisected at the center; 
        \item it divides the square into two congruent rectangles, which are,
            therefore, each half of it; 
        \item each of these rectangles is equal to one of the triangles into 
            which either diagonal divides the square.
    \end{enumerate}

    \item Let us fold the square again, laying the remaining two sides one upon
        the other. The crease now obtained and the one referred to in
        \ref{item:square} divide the square into four congruent squares.
    
    \item Folding again through the corners of the smaller squares which are at
        the centers of the sides of the larger square, we obtain a square which
        is inscribed in the latter. (Fig. 7.)
    
    \item This square is half the larger square, and has the same center.
    
    \item By joining the mid-points of the sides of the inner square, we obtain
        a square which is one-fourth of the original square (Fig. 8). By
        repeating the process, we can obtain any number of squares which are to
        one another as $$\frac{1}{2},  \frac{1}{4}, \frac{1}{8},
        \frac{1}{16}, \text{etc., or } \frac{1}{2}, \frac{1}{2^2},
        \frac{1}{2^3}, \frac{1}{2^4}, \cdots$$

        Each square is half of the next larger square, i.e., the four triangles
        cut from each square are together equal to half of it. The sums of all
        these triangles increased to any number cannot exceed the original
        square, and they must eventually absorb the whole of it.

        Therefore $$\frac{1}{2} + \frac{1}{2^2} + \frac{1}{2^3} + \text{etc.\ to
        infinite} = 1$$.


    \item The center of the square is the center of its circumscribed and
        inscribed circles. The latter circle touches the sides at their
        mid-points, as these are nearer to the center than any other points on
        the sides.

    \item Any crease through the center of the square divides it into two
        trapezoids which are congruent. A second crease through the center at
        right angles to the first divides the square into four congruent
        quadrilaterals, of which two opposite angles are right angles.  The
        quadrilaterals are concyclic, i.e., the vertices of each lie in a
        circumference.

\end{enumerate}

% i am here
%%%%%%%%%%%%%%%%%%%%%%%%%%%%%%%%%%%%%%%%%%%%%%%%%%%%%%%%%%%%%%%%%%%%%%%%%%%%%%%%
