%%%%%%%%%%%%%%%%%%%%%%%%%%%%%%%%%%%%%%%%%%%%%%%%%%%%%%%%%%%%%%%%%%%%%%%%%%%%%%%%

\chapter{THE DECAGON AND THE DODECAGON.}



\begin{enumerate}

\item Figs. 36, 37 show how a regular decagon, and a regular dodecagon, may be
    obtained from a pentagon and hexagon respectively.

%
%Fig. 36.
%
%
    The main part of the process is to obtain the angles at the center.

    In Fig. 36, the radius of the inscribed circle of the pentagon is taken for
    the radius of the circumscribed circle of the decagon, in order to keep it
    within the square.

\item A regular decagon may also be obtained as follows:

    Obtain $X$, $Y$, (Fig. 38), as in § 51, dividing $A$ in median section.

    Take $X$ the mid-point of $AB$.

    Fold $XC$, $WO$, $YD$ at right angles to $AB$.

    %Take -O in WO such that VYO=AY, or YO=XB.

%Fig. 37.


    Let $YO$, and $XO$ produced meet $XC$, and $YD$ in $C$ and $D$ respectively.

    Divide the angles $XOC$ and $DOY$ into four equal parts by $HOZL$, $KOF$, 
    and $LOG$.

    Take $OH$, $OK$, $OL$, $OF$, $OF$, and $OG$ equal to $OY$ or $OX$.

    Join $X$, $H$, $K$, $LZ$, $C$, $D$, $£$, $G$, and $Y$, in order.

    %As in § 60, LVOXS¢ of att. 736"

%Fig. 38.

    By bisecting the sides and joining the points thus determined with the
    center, the perigon is divided into sixteen equal parts. A $16$-gon is
    therefore easily constructed, and so for a $32$-gon, and in general a
    regular $2^n$-gon.
    
\end{enumerate}

%%%%%%%%%%%%%%%%%%%%%%%%%%%%%%%%%%%%%%%%%%%%%%%%%%%%%%%%%%%%%%%%%%%%%%%%%%%%%%%%
