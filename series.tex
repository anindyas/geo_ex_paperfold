%%%%%%%%%%%%%%%%%%%%%%%%%%%%%%%%%%%%%%%%%%%%%%%%%%%%%%%%%%%%%%%%%%%%%%%%%%%%%%%%

\chapter{SERIES.}


%%%%%%%%%%%%%%%%%%%%%%%%%%%%%%%%%%%%%%%%

ARITHMETIC SERIES.


\begin{enumerate}

\item Fig. 40 illustrates an arithmetic series. The horizontal lines to the left
    of the diagonal, including the upper and lower edges, form an arithmetic
    series.


%SY a ae a4 ee
%
%dS ee ee
%
%
%Fig. 40.
%
%
%The initial line being a, and d the common difference, the series is a, a+ d,
        %a+ 2d, a+ 3d, etc.
%

\item The portions of the horizontal lines to the right of the diagonal also
    form an arithmetic series, but they are in reverse order and decrease with a
    common difference.


\item In general, if / be the last term, and s the sum of the series, the above
    diagram graphically proves the formula n

%s=—5 (a+7).  105. If @ and ¢ are two alternate terms, the middle term is ate a
%

\item To insert z means between a and /, the vertical line has to be folded into
    + 1 equal parts. The common difference will be

%/—a a+1° 107. Considering the reverse series and interchanging a and /, the
        %series becomes a, @—d, @Q—2a..../.  The terms will be positive so long
        %as a>(a—1)d, and thereafter they will be zero or negative.

%%%%%%%%%%%%%%%%%%%%%%%%%%%%%%%%%%%%%%%%

GEOMETRIC SERIES.


\item In a right-angled triangle, the perpendicular from the vertex on the
    hypotenuse is a geometric mean between the segments of the hypotenuse.
    Hence, if two alternate or consecutive terms of a geometric series are
    given in length, the series can be determined as in Fig. 41. Here OP,
    OP2, OP, OPs, and OP; form a geometric series, the common rate being OP;
    : OP


%Ps
%
%
%Fig. 41.


    If OP; be the unit of length, the series consists of the natural powers of
    the common rate.


%\item Representing the series by a, ar, ar?,....  Pi Py =aV 1 +r.  P, P3=arV1+
        %7.  P3; Ps=ar*) fT + 73,
%
%
These lines also form a geometric series with the common rate ”.


\item The terms can also be reversed, in which case the common rate will be a
    proper fraction. If OP be the unit, OP; is the common rate. The sum of the
    series to infinity is OP;


%OP; — OP,

\item In the manner described in § 108, one geometric mean can be found between
    two given lines, and by continuing the process, $3$, $7$, $15$, etc., means
    can be found. In general, 2*—-1 means can be found, $m$ being any positive
    integer.


\item It is not possible to find two geometric means between two given lines,
    merely by folding through known points. It can, however, be accomplished in
    the following manner: In Fig. 41, OP; and OF, being given, it is required to
    find P, and P3. Take two rectangular pieces of paper and so arrange them,
    that their outer edges pass through 7; and /4, and two corners lie on the
    straight lines OP; and OP; in such a way that the other edges ending in
    those corners coincide.

    The positions of the corners determine OP: and OP3.


\item This process gives the cube root of a given number, for if OP, is the
    unit, the series is 1, 7, 7?, 7°.


\item There is a very interesting legend in connection with this problem. ''The
    Athenians when suffering from the great plague of eruptive typhoid fever in
    430 B.C., consulted the oracle at Delos as to how they could stop it.
    Apollo replied that they must double the size of his altar which was in the
    form of a cube.  Nothing seemed more easy, and a new altar was constructed
    having each of its edges double that of the old one.  The god, not
    unnaturally indignant, made the pestilence worse than before.  A fresh
    deputation was accordingly sent to Delos, whom he informed that it was
    useless to trifle with him, as he must have his altar exactly doubled.
    Suspecting a mystery, they applied to the geometricians.  Plato, the most
    illustrious of them, declined the task, but referred them to Euclid, who had
    made a special study of the problem.'' (Euclid's name is an interpolation
    for that of Hippocrates.)  Hippocrates reduced the question to that of
    finding two geometric means between two straight lines, one of which is
    twice as long as the other.  If a, x, y and 2a be the terms of the series,
    x?—2a*.  He did not, however, succeed in finding the means.  Menaechmus, a
    pupil of Plato, who lived between 375 and 325 B.C., gave the following three
    equations : * GP Sare Vyas ae, From this relation we obtain the following
    three equations : act soy bce Sige wie sateen (1) Pat AGe ce haeanie Mae che
    xpee Ba", is kee ha kere (3)



%*But see Beman and Smith's translation of Fink's Azstory of Mathematics, p. 82,
        %207.
%
%
%
%
%(1) and (2) are equations of parabolas and (3) is the equation of a rectangular
        %hyperbola. Equations (1) and (2) as well as (1) and (3) give «*=2a*.
        %The problem was solved by taking the intersection (@) of the two
        %parabolas (1) and (2), and the intersection () of the parabola (1) with
        %the rectangular hyperbola (3).
%
%
%* Tbid., p. 207.

%%%%%%%%%%%%%%%%%%%%%%%%%%%%%%%%%%%%%%%%

HARMONIC SERIES.


\item Fold any lines $AR$, $PBZ$, as in Fig. 42, $P$ being on $AR$, and $B$ on
    the edge of the paper. Fold again so that $AP$ and $PR$ may both coincide
    with $PB$.  Let $PX$, $PY$ be the creases thus obtained, $X$ and $VY$ being
    on $AB$.

    Then the points $A$, $X$, $B$, $VY$ form an harmonic range. That is, $AB$ is
    divided internally in $X$ and externally in $Y$ so that

%ENE Nom AeY as

    It is evident, that every line cutting $PA$, $PX$, $PB$, and $PY$ will be
    divided harmonically.


%A x B ny; Fig. 42.
%
%\item Having given 4, 4, and X, to find Y: fold any line XP and mark X
        %corresponding to B. Fold AKPR, and BP. Bisect the angle BPR by PY by
        %folding through ? so that P2 and PR coincide.
%
%Because X/ bisects the angle 4 PB,
%
%PMA AD — AP ep P.  ee Liae
%
%\item TNX hee YAS EY One —— XY XY VY =A eX Thus, 4Y, XY; and BY, are an
        %harmonic series, and XY is the harmonic mean between 4Y and BY.
        %Similarly AB is the harmonic mean between 4X and AY.

\item If $BY$ and $XY$ be given, to find the third term $AY$, we have only to
    describe any right-angled triangle on $XY$ as the hypotenuse and make angle
    $APX = \text{angle} XPB$.


%\item Let AX =2, AB= 6, and AY=e.
%
%9 Then d==———; a+¢
%
%or, ab + bc=2ac
%
%Or. eas eee ; Aho a
%
%
%When.@=6, 60.  When 6=—=2a, c= om.
%
%Therefore when X is the middle point of 44, Y is at an infinite distance to the
        %right of 2. Y approaches Bas X approaches it, and ultimately the three
        %points coincide.
%
%As X moves from the middle of AZ to the left, Y moves from an infinite distance
        %on the left towards 4, and ultimately X, 4, and Y coincide.
%
%
%\item If # be the middle point of 482, EX: EY=i2PSEE for all positions of X and
        %Y with reference to 4 or B.
%
    Each of the two systems of pairs of points $X$ and $Y$ is called a system in
    involution, the point $Z$ being called the center and $A$ or $B$ the focus
    of the system.  The two systems together may be regarded as one system.

%\item AX and AY being given, B can be found as follows:
%
%Produce XA and take AC=XA.
%
%Take D the middle point of 4 Y.
%
%Take'CE=DA or AZ=DC.
%
%F
%
%
%Fig. 43.
%
%Fold through 4 so that 4/ may be at right angles to CAY.
%
%Find / such that DF = DC.
%
%Fold through ZF and obtain 8, such that #2 is at right angles to EF.
%
%CD is the arithmetic mean between AX and AY.
%
%AF is the geometric mean between AX and AY.
%
%AF is also the geometric mean between CD or AE ~ and AB.
%
%Therefore AZ is the harmonic mean between AX and AY.
%
%\item The following is a very simple method of finding the harmonic mean
        %between two given lines.
%
%Take AB, CD on the edges of the square equal to the given lines. Fold the
        %diagonals 4D, #C and the sides AC, BD of the trapezoid ACDBZ. Fold
        %through £, the point of intersection of the diagonals, so that FEG may
        %be at right angles to the other sides of the square or parallel to 4B
        %and CD. Let KEG cut AC AL B
%
%Cc D Fig. 44.  and BD in Fand G. Then FG is the harmonic mean between AB and
        %CD.
%
%
%3 FE _ CE AB CE
%
%and £2. _ FE _ £8
%
%C. CD CB
%
%
%- FE Er. Chee
%
%' AB OD Se ae Regs. 1 1 2 ‘258 + CD FE Fe:
%
%
%Sek
%
%\item The line HK connecting the mid-points of AC and BD is the arithmetic mean
        %between 4B and CD.
%
%
%\item To find the geometric mean, take HZ in HK —FG. Fold ZM at right angles to
        %HX. Take O the mid-point of 7X and find 47 in ZM so that OM=OHL.  HM is
        %the geometric mean between 42 and CD as well as between /G and HX. The
        %geometric mean between two quantities is thus seen to be the geometric
        %mean between their arithmetic mean and harmonic mean.
%

SUMMATION OF CERTAIN SERIES,


%\item To sum the series 14+3-+45...:+(22—1).  Divide the given square into a
        %number of equal squares as in Fig. 45. Here we have 49 squares, but the
        %number may be increased as we please.
%
%The number of squares will evidently be a square number, the square of the
        %number of divisions of the sides of the given square.
%
%Let each of the small squares be considered as the unit; the figure formed by
        %4+ O-+a being called a gnomon.
%
%The numbers of unit squares in each of the gnomons AQa, BO4S, etc., are
        %respectively 3, 5, 7, 9, 11, fo.
%
%Therefore the sum of the series 1, 3, 5, 7, 9, 11, 13 is 7,
%
%Generally, 14+3+5-+.... + (2n—1)=7’.
%
%
%A B eS D 2 F 2. 3 4 S 6 7
%
%\item To find the sum of the cubes of the first x natural numbers.  Fold the
        %square into 49 equal squares as in the preceding article, and letter
        %the gnomons. Fill up the squares with numbers as in the multiplication
        %table.
%
%The number in the initial square is 1— 1°.
%
%The sums of the numbers in the gnomons 4a, Bé, etc., are 2+44 2—23, 3%, 43, 58,
        %63, and 75.
%
%The sum of the numbers in the first horizontal row is the sum of the first
        %seven natural numbers.  Let us call it s.
%
%Then the sums of the numbers in rows a, 4, ¢, d, etC., are
%
%ZEROS; 4S y OS) OSs ANGNtCs
%
%Therefore the sum of all the numbers is
%
%s1+24344454647)=s.
%
%Therefore, the sum of the cubes of the first seven natural numbers is equal to
        %the square of the sum of those numbers.
%
%Generally, 13+ 23+ 33....+ 23
%
%=(1+2+4+3....+2)%
%
%
%2 =k Sr = [OED],
%
%
%For [n-(n+1)?—[(1—1):]?  = (n? + n)? — (n? —n)? = 478.  Putting z=1, 2, 3....in
        %order, we have 4:18 = (1-2)? (0-1)?  4-23 — (2-3)? (1-2)?  4-38 —
        %(3-4)? (2-3)?
%
%
%Oe ee ee
%
%
%4-n8 —[n-(n+ 1)]?— [(#—1)-2]?.
%
%Adding we have 42n'=[n(n+1)]?
%
%
%oe ee Se
%
%
%\item If s, be the sum of the first # natural numbers,
%
%
%Re See i
%
%
%1
%
%
%\item To sum the series
%
%1:-242-343-4....+(m—l):a.
%
%In Fig. 46, the numbers in the diagonal commencing from 1, are the squares of
        %the natural numbers in order.
%
%The numbers in one gnomon can be subtracted from the corresponding numbers in
        %the succeeding gnomon. By this process we obtain n’ —(n— 1)? =
        %n?—(n—1)?
%
%+ 2[nu(m—1)+(u—2)4 (x—3)....41] =n? + (n—1)?4+2[142....4(7—1])] © =n +
        %(n—1)?+n(n—1) =[n—(u—1)]? + 3(a—1)n =1+ 3(n—1)a.
%
%Now -.: #®—(n—1)8=1-+ 3(m—l)a,
%
%wt. (2n—1)8—(a— 28 =1 4 3(n—2) (2n—1)
%
%23 —13—1+4 3-2-1 180? —1+0.  Hence, by addition, nm n+
        %3[1-2+4+2-34....+(a—1):z].
%
%Therefore
%
%
%et a eye we SS
%
%
%\item To find the sum of the squares of the first ” natural numbers.  1:24
        %2:3....+(2—l):n” = 2? 24 3?—3....+n?—n —=1?+ 2?4
        %3?....4+”7—(142+43....+7) So ae ard es gS
%
%
%Therefore
%
%
%Pe aan sd
%
%
%~_
%
%aang)" s+ 3
%
%
%__ a(n-+1) (224-1) see a cee ay
%
%
%\item To sum the series 1? + 3? + 5?....+4 (2n—1)?.  -.+ B®—(n—1)? =n? +
        %(n—1)?+ 2(n—1), by § 128, = (2n—1)?— (m—1) a, Pyopy putting *==1, 2,
        %3,..3.  13 — 0? —12— 0:1 23 18 — 37__ 1-2 33 — 23 — 522-3
%
%
%ReVativwanws ss ev
%
%
%n’ —(n—1)® = (2n—1)?—(n—1):n.
%
%
%Paap eke es!  7, j sf
%
%
%>) oe ore ee ee te a a ae 25 x“ Sy. na iw = em 7?  Ss 2 .  66 GEOMETRIC
        %EXERCISES
%
%
%Adding, we have © 3 — 124 324 52,...4 (2n—1)?  —[1-2£2-343-4....+(—1)-a].  at,
        %174-374-652... + (22—1)?  nm —n Sree | __ 4nb—n __n(2n—1)(2"+1) a
%
\end{enumerate}


%%%%%%%%%%%%%%%%%%%%%%%%%%%%%%%%%%%%%%%%%%%%%%%%%%%%%%%%%%%%%%%%%%%%%%%%%%%%%%%%
