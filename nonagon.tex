%%%%%%%%%%%%%%%%%%%%%%%%%%%%%%%%%%%%%%%%%%%%%%%%%%%%%%%%%%%%%%%%%%%%%%%%%%%%%%%%

\chapter{THE NONAGON.}

\begin{enumerate}

\item Any angle can be trisected fairly accurately by paper folding, and in this
    way we may construct approximately the regular nonagon.

%Fig. 35.
    
    Obtain the three equal angles at the center of an equilateral triangle. (§
    25.)

    For convenience of folding, cut out the three angles, $AOF$, $FOC$, and
    $COA$.  Trisect each of the angles as in Fig. 35, and make each of the arms 
    $OA$.


\item Each of the angles of a nonagon is 1,4 of a right angle — 140°.

    The angle subtended by each side at the center is 4 of a right angle or 40°.


    Half this angle is 4 of the angle of the nonagon.


\item $OA = ha$, where $a$ is the side of the square; it is also the radius of
    the circumscribed circle, 2.  The radius of the inscribed circle $= A . cos
    20°$


%= 4a cos 20° = © x0.9396926 =a X 0.4698463.  The area of the nonagon is 9 times
%the area of the triangle AOL = 9-R-4R sin 40° $ R?-sin 40°
%
%
%9a2 <0. 6427876
%
%|
%
%
%I
%
%
%—= a? 0.723136,
%
\end{enumerate}

%%%%%%%%%%%%%%%%%%%%%%%%%%%%%%%%%%%%%%%%%%%%%%%%%%%%%%%%%%%%%%%%%%%%%%%%%%%%%%%%
