%%%%%%%%%%%%%%%%%%%%%%%%%%%%%%%%%%%%%%%%%%%%%%%%%%%%%%%%%%%%%%%%%%%%%%%%%%%%%%%%

\chapter{MISCELLANEOUS CURVES.}


\begin{enumerate}

\item I propose in this, the last chapter, to give hints for tracing certain
        well-known curves.


\item This word means ivy-shaped curve. It is defined as follows: Let OQA (Fig.
        76) be a semicircle on the fixed diameter OA, and let OM, RX be two
        ordinates of the semicircle equidistant from the center. Draw OR
        cutting QW in P. Then the locus of P is the cissoid.

%If OA = 2a, the equation to the curve is
%
%y* (2a—x)=x3.
%
%Now, let PR cut the perpendicular from C in D and draw AP cutting CD in £.
%
%MV aL HO G4 iid O— PU AC, eee el OTP ee.
%
%But RV: PM=ON: OM=ON: AN=ON?: NR?
%
%== 0G42C I, ete aa Ce,
%
%If CF be the geometric mean between CD and CZ,
%
%
%*See Beman and Smith’s translation of Klein’s Famous Problems of Elementary
        %Geometry, p. 44.
%
%CD: CFC CD o- OCFECDSCD: CF=C/ CE . CD and CF are the two geometric means
        %between OC and CZ.
%
%
%Fig. 76.
%
%
%\item The cissoid was invented by Diocles (second ~ century B.C.) to find two
        %geometric means between two lines in the manner described above. OC and
        %CE being given, the point P was determined by the aid of the curve, and
        %hence the point D.
%
%
%\item If PD and DR are each equal to OQ, then the angle 4 O00 is trisected by
        %OP.
%
%Draw QR. Then QR is parallel to OA, and DQ=DP=DR=O?  ee he O==,ODCs 27 ORO=2
        %AOR:
%
%
%THE CONCHOID OR MUSSEL-SHAPED CURVE.*
%
%
%\item This curve was invented by Nicomedes (c.
%
%
%150 B.C.). Let Obea fixed point, @ its distance from a fixed line, DM, and let
        %a pencil of rays through O cut DM.  On each of these rays lay off, each
        %way from its intersection with DZ, a segment 4. The locus of the points
        %thus determined is the conchoid.  According as 46>, =, or <a, the
        %origin is a node, a cusp, or a conjugate point. The figure} represents
        %the case when >a.
%
%
%\item This curve also was employed for finding two geometric means, and for the
        %trisection of an angle.
%
%Fig. 77.
%
%
%*See Beman and Smith's translation of Klein's Famous Problems of Elementary
        %Geometry, Pp. 40.
%
%+From Beman and Smith's translation of Klein's Famous Problems of Elementary
        %Geometry, p. 46.
%
%Let OA be the longer of the two lines of which two geometric means are
        %required.  Bisect OA in B; with Oasacenter and OF asa radius describe a
        %circle. Place a chord AC in the circle equal to the shorter of the
        %given lines. Draw AC and produce 4C and BC to D and Z£, two points
        %collinear with O and such that DE=—OB, or BA.
%
%Fig. 78.
%
%
%Then #D and CZ are the two mean proportionals required.  Let OZ£ cut the
        %circles in and G.  By Menelaus’s Theorem,* BC-ED:-OA=CE:+:OD:BA « BOCAS
        %CH tes
%
%
%or BE _ OD
%
%CE ~ OA
%
%BE _OD+0A_ GE CR, OA ee
%
%
%*See Beman and Smith's New Plane and Solid Geometry, p. 240.
%
%But GE EF=>BE ‘EC.  GE ODS BA LC.  < OAS ODS=EC*.  - OA; CZL=CEL20D=0D: KC.
%
%
%The position of Z is found by the aid of the conchoid of which 4D is the
        %asymptote, O the focus, and DE the constant intercept.
%
%\item The trisection of the angle is thus effected.  In Fig. 77, let 6=/7 WOY,
        %the angle to be trisected.  On OM lay off OM=24, any arbitrary length.
        %With M as a center and a radius 4 describe a circle, and through J/
        %perpendicular to the axis of X with origin O draw a vertical line
        %representing the asymptote of the conchoid to be constructed. Construct
        %the conchoid. Connect
%    O with 4, the intersection of the circle and the conchoid. Then is /40¥Yone
        %    third of p.*
%
%
%THE WITCH.
%
%
\item If $OQA$ (Fig. 79) be a semi-circle and $VQ$ an ordinate of it, and $VP$ 
    be taken a fourth proportional to $ON$, $OA$ and $QW$, then the locus of $P$
    is the witch.

%Fold AW at right angles to OA.
%
%Fold through O, Q, and &.
%
%Complete the rectangle VAMP.
%
%PN: QN=0M:0Q2 =O0At ON,
%
%
%*Beman and Smith's translation of Klein's Famous Problems of Elementary
        %Geometry, p. 46.
%
%Therefore P is a point on the curve.  Its equation is,
%
%
%xy? =a? (a—x).
%
%
%E = ae Ms 3 fe
%
%Fig. 79.
%
%
%This curve was proposed by a lady, Maria Gaetana Agnesi, Professor of
        %Mathematics at Bologna.
%
%
%THE CUBICAL PARABOLA.  270. The equation to this curve is a?y = 2°.  Let OX and
        %OY be the rectangular axes, OA =a, andiOX == 2.  In the axis OY take
        %OB=x.  Draw 4A and draw AC at right angles to AB cutting the axis OY in
        %C.
%
%Draw CX, and draw XY at right angles to CX.  Complete the rectangle XOY.  P is
        %a point on the curve.
%
%Fig. 80.  x? x x3 Se ea eae os, Mas XP ae x res aty= x.
%
%
%THE HARMONIC CURVE OR CURVE OF SINES.
%
%
\item This is the curve in which a musical string vibrates when sounded. The
    ordinates are proportional to the sines of angles which are the same
    fractions of four right angles that the corresponding abscissas are of some
    given length.

%Let AB (Fig. 81) be the given length. Produce BA to C and fold AD perpendicular
        %to 4B. Divide the right angle DAC into a number of equal parts, say,
        %four. Mark on each radius a length equal to the amplitude of the
        %vibration, 4C=AP=AQ=AR=AD.
%
%From points ?, Q, & fold perpendiculars to 4C; then PP’, QQ’, RR’, and DA are
        %proportional to the sines of the angles PAC, QAC, RAC, DAC.
%
%Now, bisect Af in # and divide AZ and £B into twice the number of equal parts
        %chosen for the right
%
%CP’ Q’ s' Turvy
%
%
%Fig. 81.
%
%
%angle. Draw the successive ordinates SS’, 77", UU’, VV’, etc., equal to P?’,
        %GO’, RR’, DA, etc. Then S, 7, U, V are points on the curve, and V is
        %the highest point on it. By folding on VV’ and pricking through S, 7,
        %U, V, we get corresponding points on the portion of the curve VZ#. The
        %portion of the curve corresponding to #A is equal to 4 V£ but lies on
%the opposite side of 42. The length from 4 to & is half a wave length, which
        %will be repeated from Z to B on the other side of AB. LF isa point of
        %inflection on the curve, the radius of curvature there becoming
        %infinite.
%
%
%THE OVALS OF CASSINI.
%
%
\item When a point moves ina plane so that the product of its distances from two
    fixed points in the plane is constant, it traces out one of Cassini's ovals.
    The fixed points are called the foci. The equation of


%K
%
%Fig. 82.
%
%the curve is 77’ —?, where ry and *’ are the distances of any point on the
        %curve from the foci and & is a constant.
%
%Let # and F" be the foci. Fold through / and F'. Bisect FF’ in C, and fold BCS’
        %perpendicular to FF’. Find points B and J’ such that #2 and FB’ are
        %each =& Then Z and JZ’ are evidently points on
%
%
%the curve.
%
%Fold #X perpendicular to #¥’ and make (XK =—&2, and on FF” take CA and Cd’ each
        %equal to CK. Then
%
%
%A and 4’ are points on the curve.
%
%
%For CA? =CK? =CF? + FR?.  +) CA9—CF9 S89 = (C4 CF) (CA—CF) = F'A-FA,
%
%
%Produce #4 and take AT=FK. In AT takea point Mand draw WK. Fold XM’
        %perpendicular to MK meeting 4’ in J’.
%
%Then FM: FM' =’.
%
%
%With the center “and radius “, and with the center /” and radius /J/’, describe
        %two arcs cutting each other in P. Then / is a point on the curve.
%
%When a number of points between 4 and JZ are found, corresponding points in the
        %other quadrants can be marked by paper folding.
%
%When FF’ =) 2é and rr’ = 12? the curve assumes the form of a lemniscate. (§
        %279.)
%
%When F?’ is greater than }’ 24, the curve consists
%
%
%of two distinct ovals, one about each focus.
%
%
%THE LOGARITHMIC CURVE,
%
%
%\item The equation to this curve is ya‘.
%
%The ordinate at the origin is unity.
%
%If the abscissa increases arithmetically, the ordinate increases geometrically.
%
%The values of y for integral values of x can be obtained by the process given
        %in § 108.
%
%The curve extends to infinity in the angular space XOY.
%
%
%j 1 If x be negative y= a and approaches zero as x increases numerically. The
        %negative side of the axis OX is therefore an asymptote to the curve.
%
%
%THE COMMON CATENARY.
%
%
%\item The catenary is the form assumed by a heavy inextensible string freely
        %suspended from two points and hanging under the action of gravity.
%
%
%The equation of the curve is c(Z, -t J= 3 ( + é the axis of y being a vertical
        %line through the lowest point of the curve, and the axis of x a
        %horizontal line in the plane of the string at a distance ¢ below the
%
%
%lowest point; cis the parameter of the curve, and e
%
%
%the base of the natural system of logarithms.  When «==¢, y= 5 Ce! + e) when x=
        %pA Ea 5 (a + e~*) and so on.  275. From the equation y= 5 (+e >
%
%
%e can be determined graphically.
%
%
%ce—2yVetc=0
%
%
%Ve= (y+ V3?—@)
%
%CVe=yt VypP—e.  V7?— is found by taking the geometric mean between y-+ ¢c and
        %y—c.
%
%
%THE CARDIOID OR HEART-SHAPED CURVE.
%
%
%\item From a fixed point O ona circle of radius a draw a pencil of lines and
        %take off on each ray, measured both ways from the circumference, a
        %segment equal to 2a. The ends of these lines lie on a cardioid.
%
%
%ete A Fig. 83.  The equation to the curve is y=a(1-+ cos@).  The origin is a
        %cusp on the curve. The cardioid is the inverse of the parabola with
        %reference to its
%
%
%focus as center of inversion.
%
%
%THE LIMACON,
%
%
%\item From a fixed point ona circle, draw a number of chords, and take off a
        %constant length on each of these lines measured both ways from the
        %circumference of the circle.
%
%If the constant length is equal to the diameter of the circle, the curve is a
        %cardioid.
%
%If it be greater than the diameter, the curve is altogether outside the circle.
%
%If it be less than the diameter, a portion of the curve lies inside the circle
        %in the form of a loop.
%
%If the constant length is exactly half the diameter, the curve is called the
        %trisectrix, since by its aid any angle can be trisected.
%
%
%The equation is y—acos 6+ 4.
%
%The first sort of limacon is the inverse of an ellipse ; and the second sort is
        %the inverse of an hyperbola, with reference to a focus as a center. The
        %loop is the inverse of the branch about the other focus.
%
%
%\item The trisectrix is applied as follows:
%
%Let 4OB be the given angle. Take O4, OF equal to the radius of the circle.
        %Describe a circle with the center O and radius OA or OB. Produce AO in-
        %definitely beyond the circle. Apply the trisectrix so that O may
        %correspond to the center of the circle and OB the axis of the loop. Let
        %the outer curve cut 40 produced in C. Draw AC cutting the circle in D,
        %Draw OD.
%
%Fig. 85.
%
%
%Then /ACB istof 7 AOB.  For CD=DO0=O08, LZAOB=ZACB+ CBO =<=/ACB+/ODB
        %=/ACB-+2/ACB =3 /7ACB.
%
%
%THE LEMNISCATE OF BERNOULLI.
%
%
%\item The polar equation to the curve is
%
%r? =a’ cos 26.  Let O be the origin, and OA =a.  Produce AO, and draw OD at
        %right angles to OA Take the angle 4OP=@ and AOB=28.  Draw AB
        %perpendicular to O#.  In AO produced take OC=OB.
%
%Find D in OD such that CDA is a right angle.  aiken fea).  FP is a point on the
        %curve.  P=0Ir=06*04 —=OLROF =acos26-a =a’ cos26.
%
%As stated above, this curve is a particular case of the ovals of Cassini.
%
%Fig. 86.
%
%
    It is the inverse of the rectangular hyperbola, with reference to its center
    as center of inversion, and also its pedal with respect to the center.


%The area of the curve is a?.
%
%THE CYCLOID,

\item The cycloid is the path described by a point on the circumference of a
    circle which is supposed to roll upon a fixed straight line.

    Let $A$ and $A'$ be the positions of the generating point when in contact
    with the fixed line after one complete revolution of the circle.  Then $AA'$
    is equal to the circumference of the circle.

    The circumference of a circle may be obtained in length in this way.  Wrap a
    strip of paper round a circular object, e.g., the cylinder in Kindergarten
    gift No. II., and mark off two coincident points.  Unfold the paper and
    fold through the points. Then the straight line between the two points is
    equal to the circumference corresponding to the diameter of the cylinder.


    By proportion, the circumference corresponding to any diameter can be found
    and vice versa.

%Fig. 87.
%
%Bisect 4/4’ in D and draw D&# at right angles to AA', and equal to the diameter
        %of the generating circle.
%
%Then 4, A’ and BZ are points on the curve.
%
%Find O the middle point of BD.
%
%Fold a number of radii of the generating circle through O dividing the
        %semi-circumference to the right into equal arcs, say, four.
%
%Divide AD into the same number of equal parts.
%
%Through the ends of. the diameters fold lines at right angles to BD.
%
%Let ZF P be one of these lines, / being the end of a radius, and let G be the
        %corresponding point of section of dD, commencing from D. Mark off /P
        %equal to GA or to the length of arc BF.
%
%Then / is a point on the curve.

    Other points corresponding to other points of section of $AD$ may be marked
    in the same way.

    The curve is symmetric to the axis $BD$ and corresponding points on the
    other half of the curve can be marked by folding on $BD$.

    The length of the curve is 4 times $BD$ and its area 3 times the area of the
    generating circle.


%THE TROCHOID.

\item If as in the cycloid, a circle rolls along a straight line, any point in
    the plane of the circle but not on its circumference traces out the curve
    called a trochoid.


%THE EPICYCLOID.

\item An epicycloid is the path described by a point on the circumference of a
    circle which rolls on the circumference of another fixed circle touching it
    on the outside.

%THE HYPOCYCLOID.

\item If the rolling circle touches the inside of the fixed circle, the curve
    traced by a point on the circumference of the former is a hypocycloid.

    When the radius of the rolling circle is a sub-multiple of the fixed circle,
    the circumference of the latter has to be divided in the same ratio.

    These sections being divided into a number of equal parts, the position of
    the center of the rolling circle and of the generating point corresponding
    to each point of section of the fixed circle can be found by dividing the
    circumference of the rolling circle into the same number of equal parts.


%THE QUADRATRIX.*

\item Let $OACB$ bea square. If the radius $OA$ of a circle rotate uniformly
    round the center O from the position $OA$ through a right angle to $OF$ and
    if in the same time a straight line drawn perpendicular to $OB$ move
    uniformly parallel to itself from the position $OA$ to $BC$; the locus of
    their intersection will be the quadratrix.

    This curve was invented by Hippias of Elis (420 B.C.) for the multisection
    of an angle.  If $P$ and $P'$ are points on the curve, the angles $AOP$ and
    $AOP'$ are to one another as the ordinates of the respective points.


%THE SPIRAL OF ARCHIMEDES.

\item If the line $OA$ revolve uniformly round $O$ as center, while point $P$
    moves uniformly from O along $OA$, then the point $X$ will describe the
    spiral of Archimedes.

%*Beman and Smith's translation of Klein’s Famous Problems of Elementary
        %Geometry, Pp. 57.

\end{enumerate}

%%%%%%%%%%%%%%%%%%%%%%%%%%%%%%%%%%%%%%%%%%%%%%%%%%%%%%%%%%%%%%%%%%%%%%%%%%%%%%%%
