%%%%%%%%%%%%%%%%%%%%%%%%%%%%%%%%%%%%%%%%%%%%%%%%%%%%%%%%%%%%%%%%%%%%%%%%%%%%%%%%

\chapter{THE CONIC SECTIONS.}


\section{SECTION I.—THE CIRCLE.}


\begin{enumerate}


\item A piece of paper can be folded in numerous ways through a common point.
        Points on each of the lines so taken as to be equidistant from the
        common point will lie on the circumference of a circle, of which the
        common point is the center. The circle is the locus of points
        equidistant from a fixed point, the centre.


\item Any number of concentric circles can be drawn. They cannot meet each
    other.


\item The center may be considered as the limit of concentric circles described
    round it as center, the radius being indefinitely diminished.


\item Circles with equal radii are congruent and equal.


\item The curvature of a circle is uniform throughout the circumference. A
    circle can therefore be made to slide along itself by being turned about its
    center.  Any figure connected with the circle may be turned about the center 
    of the circle without changing its relation to the circle.

\item A straight line can cross a circle in only two points.

\item Every diameter is bisected at the center of the circle. It is equal in
    length to two radii. All diameters, like the radii, are equal.


\item The center of a circle is its center of symmetry, the extremities of any
    diameter being corresponding points.


\item Every diameter is an axis of symmetry of the circle, and conversely.

\item The propositions of §§ 188, 189 are true for systems of concentric
    circles.

\item Every diameter divides the circle into two equal halves called
    semicircles.

\item Two diameters at right angles to each other divide the circle into four
    equal parts called quadrants. 

\item By bisecting the right angles contained by the diameters, then the half
    right angles, and so on, we obtain 2” equal sectors of the circle. The angle
    ie so : between the radii of each sector is =— of a right angle 2a Ls =

\item As shown in the preceding chapters, the right angle can be divided also
    into $3$, $5$, $9$, $10$, $12$, $15$ and $17$ equal parts. And each of the
    parts thus obtained can be subdivided into $2^n$ equal parts.

\item A circle can be inscribed in a regular polygon, and a circle can also be
    circumscribed round it. The former circle will touch the sides at their
    mid-points.

\item Equal arcs subtend equal angles at the center; and conversely. This can be
    proved by superposition. If a circle be folded upon a diameter, the two
    semi-circles coincide. Every point in one semi-circumference has a
    corresponding point in the other, below it.

\item Any two radii are the sides of an isosceles triangle, and the chord which
    joins their extremities is the base of the triangle.

\item A radius which bisects the angle between two radii is perpendicular to the
    base chord and also bisects it.


\item Given one fixed diameter, any number of pairs of radii may be drawn, the
    two radii of each set being equally inclined to the diameter on each side of
    it.  The chords joining the extremities of each pair of radii are at right
    angles to the diameter.  The chords are all parallel to one another.

\item The same diameter bisects all the chords as well as arcs standing upon the
    chords, i. e., the locus of the mid-points of a system of parallel chords is
    a diameter.

\item The perpendicular bisectors of all chords of a circle pass through the
    center.

\item Equal chords are equidistant from the center.

\item The extremities of two radii which are equally inclined to a diameter on
    each side of it, are equidistant from every point in the diameter.  Hence,
    any number of circles can be described passing through the two points. In
    other words, the locus of the centers of circles passing through two given
    points is the straight line which bisects at right angles the join of the
    points.

\item Let $CC'$ be a chord perpendicular to the radius $OA$.  Then the angles
    $AOC$ and $AOC$' are equal.  Suppose both move on the circumference towards
    $A$ with the same velocity, then the chord $CC''$ is always parallel to
    itself and perpendicular to $OA$.  Ultimately the points $C$, $A$ and $C''$
    coincide at $A$, and $CAC$' is perpendicular to $OA$.  $A$ is the last point
    common to the chord and the circumference.  $CAC'$ produced becomes
    ultimately a tangent to the circle.

\item The tangent is perpendicular to the diameter through the point of contact;
    and conversely.


\item If two chords of a circle are parallel, the arcs joining their extremities
    towards the same parts are equal. So are the arcs joining the extremities of
    either chord with the diagonally opposite extremities of the other and
    passing through the remaining extremities.  This is easily seen by folding
    on the diameter perpendicular to the parallel chords.

\item The two chords and the joins of their extremities towards the same parts
    form a trapezoid which has an axis of symmetry, viz., the diameter
    perpendicular to the parallel chords. The diagonals of the trapezoid
    intersect on the diameter. It is evident by folding that the angles between
    each of the parallel chords and each diagonal of the trapezoid are equal.

    Also the angles upon the other equal arcs are equal.


\item The angle subtended at the center of a circle by any arc is double the
    angle subtended by it at the circumference.  

%Vv P| Pe AX ; A X X Fig. 61. Fig. 62, Fig. 63.


    An inscribed angle equals half the central angle standing on the same arc.
    Given $AVB$ an inscribed angle, and $AOBA$ the central angle on the same arc
    $AB$.  To prove that %7 $AVB$=}3/$AOB$.  

    Proof.  1. Suppose VO drawn through center O, and produced to meet the
    circumference at X.

%mnens /XVi==] 7 VBO: 2. And LXOB= /XVB4+ 4 VBO, SBR AVE.  3. on ZAV Boe 7 XO.
        %4, Similarly 7 4 VX=1/ AOX (each=zero in Fig. 62),
%
%Suds! 24 rh 408,
%
%The proof holds for all three figures, point 4 having moved to X (Fig. 62), and
        %then through X (Fig.  63).*
%
%\item The angle at the center being constant, the angles subtended by an arc at
        %all points of the cir-
%
%
%cumference are equal.  210. The angle in a semicircle is a right angle.
%
%
%2u1. If dB be a diameter of a circle, and DC a chord at right angles to it,
        %then 4C&D is a quadrilateral of which 4# is an axis of symmetry. The
        %angles BCA and ADB being each a right angle, the remaining two angles
        %DBC and CAD are together equal toa straight angle. If 4’ and ZB’ be any
        %other points on the arcs DAC and C&D respectively, the Z CAD=/CGA'Dand
        %/
%DBC=ZDBP'C, and / CA'D + D£&’'C =a straight angle. Therefore, also, / B’CA’ +
        %/A'DB' =a straight angle.
%
%Conversely, if a quadrilateral has two of its opposite angles together equal to
        %two right angles, it is inscriptible in a circle.
%
%
%*The above figures and proof are from Beman and Smith's New Plane and Solid
        %Geometry, p. 129.

\item The angle between the tangent to a circle and a chord which passes through
    the point of contact is equal to the angle at the circumference standing
    upon that chord and having its vertex on the side of it opposite to that
    on which the first angle lies.

    Let $AC$ be a tangent to the circle at $A$ and $AB$ a chord. Take $O$ the
    center of the circle and draw $OA$, $OB$. Draw $OD$ perpendicular to $AB$.

%Then /BAC=/ AOD=}2 BOA.
%
%Fig. 64.
%

\item Perpendiculars to diameters at their extremities touch the circle at these
    extremities. (See Fig.64).


    The line joining the center and the point of intersection of two tangents
    bisects the angles between the two tangents and between the two radii.  It
    also bisects the join of the points of contact. The tangents are equal.

    This is seen by folding through the center and the point of intersection of
    the tangents.

    Let $AC$, $AB$ be two tangents and $ADZLOF$ the line through the 
    intersection of the tangents $A$ and the center $O$, cutting the circle in
    $D$ and $Z$ and $BC$ mn 2.

    Then $AC$ or $AB$ is the geometric mean of $AD$ and $AF$; $AE$ is the
    harmonic mean; and $AO$ the arithmetic mean.

%
%
%Bi = AD AF, Apes OA-AL; ' Apa 1P AF 24D: AF OA ADI AF
%
%Similarly, if any other chord through 4 be obtained cutting the circle in P and
    %& and BC in Q, then 4Q is the harmonic mean and AC the geometric mean
    %between 4/ and AR.
%
%
%\item Fold a right-angled triangle OCB and CA the perpendicular on the
    %hypotenuse. Take Din AB such that OD= OC (Fig. 65).
%
%
%Then 04:O28=0OC?*= OD’, ance C= OG OF, CA aes OL OB.
%
%A circle can be described with O as center and OC or ODP as radius.
%
%The points 4 and B are inverses of each other with reference to the center of
    %inversion O and the circle of inversion CDE.
%
%Fig. 65.
%
%
%Hence when the center is taken as the origin, the foot of the ordinate of a
    %point on a circle has for its inverse the point of intersection of the
    %tangent and the axis taken.
%
%
%\item Fold “BG perpendicular to OZ. Then the line “BG is called the polar of
    %point 4 with reference to the polar circle CD and polar center O; and 4 is
    %called the pole of “2G. Conversely B is the pole of
%
%CA and CA is the polar of # with reference to the
%
%
%same circle.
%
%
%\item Produce OC to meet “BG in F, and fold AH perpendicular to OC.
%
%Then “and / are inverse points.
%
%Aff is the polar of /, and the perpendicular at / to OF is the polar of ZH.
%
%
%\item The points 4, 5, /, #, are concyclic.
%
%That is, two points and their inverses are concyclic; and conversely.
%
%Now take another point G on FBG. Draw OG, and fold AX perpendicular to OG. Then
    %X andG
%
%
%are inverse points with reference to the circle CDZ.
%
%
%\item The points /, B, G are collinear, while their polars pass through 4.
    %Hence, the polars of collinear points are concur-
%
%
%rent.
%
%
%\item Points so situated that each lies on the polar of the other are called
    %conjugate points, and lines so related that each passes through the pole of
    %the other are called conjugate lines.
%
%A and Fare conjugate points, so are 4 and B, A and G.
%
%The point of intersection of the polars of two points is the pole of the join
    %of the points.
%
%
%\item As 4 moves towards D, B also moves up to it.  Finally 4 and & coincide
    %and “BG is the tangent at B.
%
%Hence the polar of any point on the circle is the
%
%
%tangent at that point.
%
%
%\item As 4 moves back to O, B moves forward to infinity. The polar of the
    %center of inversion or the
%
%
%polar center is the line at infinity.
%
%
\item The angle between the polars of two points is equal to the angle subtended
    by these points at the polar center.


\item The circle described with 2 as a center and BC as a radius cuts the circle
    CDE orthogonally.


\item Bisect $AB$ in $Z$ and fold $ZN$ perpendicular to $AB$. Then all circles
    passing through $A$ and $B$ will have their centers on this line.  These
    circles cut the circle $CDZ$ orthogonally.  The circles circumscribing the
    quadrilaterals $ABFH$ and $ABGK$ are such circles.  $AF$ and $AG$ are
    diameters of the respective circles.  Hence if two circles cut orthogonally
    the extremities of any diameter of either are conjugate points with respect
    to the other.


\item The points O, A, H and X are concyclic. #, A, K being inverses of points on
    the line /7G, the inverse of a line is a circle through the center of
    inversion and the pole of the given line, these points being the extremities
    of a diameter; and conversely.


\item If DO produced cuts the circle CDZ in JD’, D and J’ are harmonic
    conjugates of 4 and 2. Similarly, if any line through B cuts AC in A’ and
    the circle CDE ind and a’, then d and a’ are harmonic conjugates of 4’ and
    2.


\item Fold any line ZWV=LZB=ZA, and MO’ perpendicular to ZM meeting AB produced
    in $O'$.

    Then the circle described with center $O'$ and radius $O'A/$ cuts
    orthogonally the circle described with center $Z$ and radius $ZAZ$.

%Now OM =O + LE, and OL?=OM* + LM?.  : O2—COL?= O0F—OM?.
%
%.‘. ZMis the radical axis of the circles O (OC) and O'(0'M).

    By taking other points in the semi-circle 4MB and repeating the same
    construction as above, we get two infinite systems of circles co-axial with
    O(OC) and O'(O'M), viz., one system on each side of the radical axis, ZV.
    The point circle of each system is a point, A or B, which may be regarded
    as an infinitely small circle.

    The two infinite systems of circles are to be regarded as one co-axial
    system, the circles of which range from infinitely large to infinitely
    small—the radical axis being the infinitely large circle, and the limiting
    points the infinitely small. This system of co-axial circles is called the
    limiting point species.

    If two circles cut each other their common chord is their radical axis.
    Therefore all circles passing through $A$ and $B$ are co-axial. This system
    of co-axial circles is called the common point species.


\item Take two lines $OAB$ and $OPQ$.  From two points $A$ and $B$ in $OAB$ draw
    $AP$, $BQ$ perpendicular to $OPQ$.  Then circles described with $4A$ and $B$
    as centers and $47$ and $BQ$ as radii will touch the line $OPQ$ at $P$ and
    $Q$.

%Then O4 SCOR = AP TSG:

    This holds whether the perpendiculars are towards the same or opposite
    parts.  The tangent is in one case direct, and in the other transverse.

    In the first case, O is outside 4A, and in the second it is between 4 and Z.
    In the former it is called the external center of similitude and in the
    latter the internal centre of similitude of the two circles.


\item The line joining the extremities of two parallel radii of the two circles
    passes through their external center of similitude, if the radii are in the
    same direction, and through their internal center, if they are drawn in
    opposite directions.


\item The two radii of one circle drawn to its points of intersection with any
    line passing through either center of similitude, are respectively parallel
    to the two radii of the other circle drawn to its intersections with the
    same line.

%\item All secants passing through a center of similitude of two circles are cut
    %in the same ratio by the circles.
%
%232.. If Ai, Di, and £2, Dz be the points of intersection, 2y;, Bo, and Dy, D2
    %being corresponding points,
%
%A1C1 XoCy
%
%Hence the inverse of a circle, not through the center of inversion is a circle.
%
%
%OP: OD = OD: COf5— OC?
%
%
%
%Fig. 66.

    The center of inversion is the center of similitude of the original circle
    and its inverse.  The original circle, its inverse, and the circle of
    inversion are co-axial.


\item The method of inversion is one of the most important in the range of
    Geometry.  It was discovered jointly by Doctors Stubbs and Ingram, Fellows
    of Trinity College, Dublin, about 1842.  It was employed by Sir William
    Thomson in giving geometric proof of some of the most difficult propositions
    in the mathematical theory of electricity.


%%%%%%%%%%%%%%%%%%%%%%%%%%%%%%%%%%%%%%%%


\section{SECTION II.—THE PARABOLA.}


\item A parabola is the curve traced by a point which moves in a plane in such a
    manner that its distance from a given point is always equal to its distance
    from a given straight line.


\item Fig. 67 shows how a parabola can be marked on paper. The edge of the
    square $JZV$ is the directrix, $O$ the vertex, and $X$ the focus. Fold
    through $OX$ and obtain the axis. Divide the upper half of the square into a
    number of sections by lines parallel to the axis. These lines meet the
    directrix in a number of points. Fold by laying each of these points on the
    focus and mark the point where the corresponding horizontal line is cut. The
    points thus obtained lie ona parabola. The folding gives also the tangent to
    the curve at the point.

%Fig. 67.
%

\item $BZ$ which is at right angles to $OX$ is called the semi-latus rectum.


\item When points on the upper half of the curve have been obtained,
    corresponding points on the lower half are obtained by doubling the paper on
    the axis and pricking through them.


\item When the axis and the tangent at the vertex are taken as the axes of
    co-ordinates, and the vertex as origin, the equation of the parabola
    becomes

%
%y?=4ax or PN?—4:-OF- ON.
%
%Fig. 68.

    The parabola may be defined as the curve traced by a point which moves in
    one plane in such a manner that the square of its distance from a given
    straight line varies as its distance from another straight line; or the
    ordinate is the mean proportional between the abscissa, and the latus rectum
    which is equal to $A-OF$. Hence the following construction.

%Take OT in FO produced —4:- OF.
%
%Bisect ZW in M.
%
%Take Q in OY such that VO=MN=MT.
%
%Fold through Q so that QP may be at right angles to OY.
%
%Let P be the point where QP meets the ordinate of WV.
%
%
%Then / is a point on the curve.
%
%
%\item The subnormal —=20F and FP=FG=FT".
%
%These properties suggest the following construction.
%
%Take JV any point on the axis.
%
%On the side of V remote from the vertex take NG=20F.
%
%Fold VP perpendicular to OG and find Pin VP such that FP= FG.
%
%Then / is a point on the curve.
%
%A circle can be described with F as center and 7G, FP and F7" as radii.
%
%The double ordinate of the circle is also the double ordinate of the parabola,
    %i. e., P describes a parabola
%
%
%as /V moves along the axis.
%
%
%\item Take any point VV’ between O and F (Fig. 69).  Fold RN?’ at right angles
    %to OF.
%
%Take & so that OR= OF.
%
%Fold RN perpendicular to OX, WV being on the axis.
%
%Fold VP perpendicular to the axis.  Now, in OX take OT=OW'’.
%
%Take 7’ in RN' so that FP’ = FT.  Fold through P’F cutting VP in P.  Then 7 and
    %/ are points on the curve.
%
%Fig. 69.
%
%
%\item Vand NW’ coincide when PF?’ is the latus rectum. 5
%
%As NV’ recedes from F to O, V moves forward from F to infinity.
%
%At the same time, Z moves toward O, and 7"(O7’=
%
%
%ON) moves in the opposite direction toward infinity.


\item To find the area of a parabola bounded by the axis and an ordinate.
    Complete the rectangle OVPK. Let OX be divided into z equal portions of
    which suppose Om to contain » and mz to be the $(r+ 1)''$. Draw mp, ng at
    right angles to OX meeting the curve in 4, g, and px’ at right angles to mg.
    The curvilinear area OPX is the limit of the sum of the series of rectangles
    constructed as mn’ on the portions corresponding to mn.  But Cof4:
    —WVK=pm-mn: PK: OR, and, by the properties of the parabola, ppi PK == O08
    OK?  a7? 2 7?  and mnu:OK=1:n.  ae pm-mn:PK: OX=?*> nt.  copa XK Hence the
    sum of the series of rectangles re 2+ eg + (n—1)? x ONE a 2n— Be ai 2n3 —3n?
    +n 1:2°3-m8

%XxOVE 1 1 1 < =(;-%+t gna) X NE = 41 of CK in the limit, i. e., when # is o.
    %.‘. The curvilinear area OPK =} of (1K, and the parabolic area OPN =2 of
    %MWK.
%
%
%\item The same line of proof applies when any diameter and an ordinate are
    %taken as the boundaries of the parabolic area.
%
%%%%%%%%%%%%%%%%%%%%%%%%%%%%%%%%%%%%%%%%


\section{SECTION III.—THE ELLIPSE.}

\item An ellipse is the curve traced by a point which moves in a plane in such a
    manner that its distance from a given point is in aconstant ratio of less
    inequality to its distance from a given straight line.


%Fig. 70.
    Let $X$ be the focus, $OY$ the directrix, and $XX$’ the perpendicular to
    $OY$ through.  Let $FA$: $AO$ be the constant ratio, #A being less than
    $AO$.  $A$ is a point on the curve called the vertex.  As in § 116, find
    $A'$ in $XX$’ such that $FATA$ O= $FA$: $AO$.


    Then $A'$ is another point on the curve, being a second vertex.

    Double the line $AA'$ on itself and obtain its middle point $C$, called the
    center, and mark /” and $O'$ corresponding to $F$ and $O$.  Fold through
    $O'$ so that $O'Y'$ may be at right angles to $XX'$. Then /” is the second
    focus and $O'Y'$ the second directrix.


%By folding 4 4’, obtain the perpendicular through C.
%FA:AO=FA':A'O — F444 FA’: AO+A'O ad's OF =CAl CO,

    Take points $B$ and $JZ'$ in the perpendicular through $C$ and on opposite
    sides of it, such that $XB$ and $FZ'$ are each equal to $CA$. Then # and Z’
    are points on the curve.

    $AA'$ is called the major axis, and $AB'$ the minor axis.


\item To find other points on the curve, take any point # in the directrix, and
    fold through $Z$ and $A$, and through $Z$ and $A'$.  Fold again through &
    and & and mark the point $P$ where $FA'$ cuts $ZA$ produced.  Fold through
    $P¥$ and $P'$ on $ZA$'.  Then $P$ and $P'$ are points on the curve.

%Fold through P and /’ so that APZ and A'L'/" are perpendicular to the
    %directrix, A and A’ being on the directrix and Z and Z’ on EZ.
%
%FTL bisects the angle 4'FP,
%
%
%ot. (LEFP=/ PEF and FP=FEI.  FP:PK=PL: PK =FA: AO.
%
%
%And PPP KS FL PK
%
%ed een == A AO: If EO—FO, FP is at right angles to FO, and FP=FP',. PP’ is the
    %latus rectum.


\item When a number of points on the left half of the curve are found,
    corresponding points on the other half can be marked by doubling the paper
    on the minor axis and pricking through them.

\item An ellipse may also be defined as follows :
    If a point $P$ move in such a manner that $P/VP$  $AN$ $VA'$ is a constant
    ratio, $PV$ being the distance of P from the line joining two fixed points
    $A$, $A'$, and $JV$ being between A and J', the locus of P is an ellipse of
    which $AA'$ is an axis.


%\item In the circle, PV? = AN: NA’.
%In the ellipse PV?: 44-4’ is a constant ratio.

    This ratio may be less or greater than unity. In the former case / APA’ is
    obtuse, and the curve lies within the auxiliary circle described on 44’ as
    diameter. Inthe latter case, / APA’ is acute and the curve is outside the
    circle. In the first case AA’ is the major, and in the second it is the
    minor axis.


\item The above definition corresponds to the equation B2 gt eee Z (2ax — x?)
    when the vertex is the origin.

%\item AV: V4' is equal to the square on the ordinate QV of the auxiliary
    %circle, and PV:Q0N= Faas

%Fig. 71.

\item Fig. 71 shows how the points can be determined when the constant ratio is
    less than unity.  Thus, lay off CD—AC, the semi-major axis. Through £ any
    point of dC draw DZ ard produce it to meet the auxiliary circle in Q@. Draw
    A'£ and produce it to meet the ordinate QV in P. Then is PV: ON = BC:
    DC=BC:AC. The same process is applicable when the ratio is greater than
    unity. When points in one quadrant are found, corresponding points in other
    quadrants can be easily marked.

\item If P and /” are the extremities of two conjugate diameters of an ellipse
    and the ordinates WP and J/'/” meet the auxiliary circle in Q and Q’, the
    angle QCQ' is a right angle.

    Now take a rectangular piece of card or paper and mark on two adjacent edges
    beginning with the common corner lengths equal to the minor and major axes.
    By turning the card round C mark corresponding points on the outer and inner
    auxiliary circles.  Let Q, \& and Q’, \&’ be the points in one position.
    Fold the ordinates QM and Q’M’, and RP and FP’, perpendiculars to the
    ordinates.  Then P and /” are points on the curve.

%Fig. 72.

\item Points on the curve-may also be easily determined by the application of
    the following property of the conic sections.

    The focal distance of a point on a conic is equal to the length of the
    ordinate produced to meet the tangent at the end of the latus rectum.

\item Let $A$ and $A'$ be any two points. Draw $AA'$ and produce the line both
    ways.  From any point $D$ in $A'A$ produced draw $DA$ perpendicular to dD.
    Take any point $R$ in $DR$ and draw $RA$ and $RA'$.  Fold $AP$ perpendicular
    to $A2$, meeting $RA'$ in $P$.  For different positions of $R$ in $DR$, the
    locus of $P$ is an ellipse, of which $AA'$ is the major axis.

%Fig. 73.

    Fold $PX$ perpendicular to $AA'$.

    Now, because $PJ$ is parallel to $RD$, PN: A'N=RD: A'D.  Again, from the
    triangles, dP and DAR, PN: AN=AD: RD, .-. PN?:AN:+A'N=AD:A'D, a constant
    ratio, less than unity, and it is evident from the construction that VV must
    lie between $A$ and $A'$.



%%%%%%%%%%%%%%%%%%%%%%%%%%%%%%%%%%%%%%%%


\section{SECTION IV.—THE HYPERBOLA.}


\item An hyperbola is the curve traced by a point which moves in a plane in such
    a manner that its distance from a given point is in a constant ratio of
    greater inequality to its distance from a given straight line.

\item The construction is the same as for the ellipse, but the position of the
    parts is different. As explained in § 119, X, 4’ lies on the left side of
    the directrix. Each directrix lies between 4 and 4’, and the foci lie
    without these points. The curve consists of two branches which are open on
    one side.  The branches lie entirely within two vertical angles formed by
    two straight lines passing through the center which are called the
    asymptotes. These are tangents to the curve at infinity.


\item The hyperbola can be defined thus: If a point fF move in such a manner
    that PV?:AN-NA' isa constant ratio, PV being the distance of P from the line
    joining two fixed points 4 and 4’, and J not being between 4 and 4’, the
    locus of P is an hyperbola, of which 44’ is the transverse axis.

    This corresponds to the equation tes 2 J (Zax x*), where the origin is at
    the right-hand vertex of the hyperbola.

%Fig. 74 shows how points on the curve may be found by the application of this
    %formula.
%
%Let C be the center and 4 the vertex of the curve.
%
%Gi = Ch =s; CA’ =CA=CA =a.
%
%Fold CD any line through C and make CD=CA.  Fold DJ perpendicular to CD. Fold
    %VQ perpendicular to C4 and make VQ=DJWN. Fold QA” cutting CA in S. Fold A’S
    %cutting QV in P.
%
%Fig. 74.
%
%
%Then / is a point on the curve.  For, since DW is tangent to the circle on the
    %diameter 4’A DN*=AN:(2CA+AN), or since QON=DN, QN? = x(2a-+ x).
%
%ON ATE LAN Be : x(2a-+ x) a Squaring, Re may area es 2 or y = Ga (Aan + x”).
%
%
%If 0V=é4 then ZW is the focus and CD is one of the asymptotes. If we complete
    %the rectangle on AC and BC the asymptote is a diagonal of the rectangle.
%
%
%\item The hyperbola can also be described by the property referred to in § 253.
%
%
%\item An hyperbola is said to be equilateral when the transverse and conjugate
    %axes are equal. Here a=4, and the equation becomes
%
%y? = (2a+ x)x.
%
%In this case the construction is simpler as the ordinate of the hyperbola is
    %itself the geometric mean between AW and 4’J, and is therefore equal to the
    %tangent from to the circle described on 4 4’ as diameter.
%
%\item The polar equation to the rectangular hyperbola, when the center is the
    %origin and one of the axes the initial line, is
%
%r? cos 20=a?  a cos20 ”*
%
%Let OX, OY be the axes; divide the right angle YOX into a number of equal
    %parts.  Let XOA, AOB be two of the equal angles. Fold XB at right angles to
    %OX. Produce BO and take OF=OX. Fold OG perpendicular to BF and find G in OG
    %such that FGB is aright angle. Take O4 = OG. Then 4 isa point on the curve.
%
%OL 72 =—
%
%
%
%
%Fig. 75.
%
%
%Now, the angles XOA and AOB being each 6, a OES ae And 042= 0G?= OB -OF=—*, a.
    %cos 26
%
%
%. cos20=—a’.
%
%
%\item The points of trisection of a series of conterminous circular arcs lie on
    %branches of two hyperbolas of which the eccentricity is 2. This theorem
    %affords a means of trisecting an angle. *
%
%
%*See Taylor's Ancient and Modern Geometry of Conics, examples 308, 390 with
    %footnote,
%
\end{enumerate}

%%%%%%%%%%%%%%%%%%%%%%%%%%%%%%%%%%%%%%%%%%%%%%%%%%%%%%%%%%%%%%%%%%%%%%%%%%%%%%%%
